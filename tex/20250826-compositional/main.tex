\documentclass{article}

\usepackage[pdftex,usenames,dvipsnames]{xcolor}
\usepackage{mathpartir}
\usepackage{amsmath}
\usepackage{amsthm}
\usepackage{amssymb}
\usepackage{stmaryrd}
\usepackage{placeins}
\usepackage{mathtools}



\newcommand{\thecalculus}{\textsc{BGV}\xspace}

\newcommand{\metacolor}{\textcolor{black}}
\newcommand{\ctxcolor}{\textcolor{RoyalPurple}}
\newcommand{\targetctxcolor}{\textcolor{blue}}
\newcommand{\typecolor}{\textcolor{RoyalPurple}}
\newcommand{\targettypecolor}{\textcolor{blue}}
\newcommand{\termcolor}{\textcolor{Mahogany}}
\newcommand{\priocolor}{\textcolor{Emerald}}  % LimeGreen -> too light
\newcommand{\dircolor}{\textcolor{Melon}}
\newcommand{\mobcolor}{\textcolor{LimeGreen}}
\newcommand{\targettermcolor}{\textcolor{Sepia}}

\newcommand*\Keyword[1]{\operatorname{\texttt{#1}}}
\newcommand*\TermKeyword[1]{\termcolor{\Keyword{#1}}}
\newcommand*\TypeKeyword[1]{\typecolor{\Keyword{#1}}}
\newcommand*\TargetTypeKeyword[1]{\targettypecolor{\Keyword{#1}}}
\newcommand*\TargetTermKeyword[1]{\targettermcolor{\Keyword{#1}}}

% Grammars
\newcommand{\grmeq}{\; \Coloneqq \;\;}
\newcommand{\grmor}{\;\mid\;}

% Mobilities
\newcommand*\Mob{\mobcolor{m}}
\newcommand*\MobMobile{\mobcolor{\textsc m}}
\newcommand*\MobStationary{\mobcolor{\textsc s}}

% Multiplicities
\newcommand*\Mul{m}
\newcommand*\MulLin{1}
\newcommand*\MulUnr{\omega}

% Directions
\newcommand*\Dir{\dircolor{d}}
\newcommand*\DirLeft{\dircolor{\textsc l}}
\newcommand*\DirRight{\dircolor{\textsc r}}
\newcommand*\DirNone{\dircolor{\omega}} % {\cdot} % This should be un
\newcommand*\DirUnord{\dircolor{\mathbf 1}} % {\circ} %This should be lin

% Pairs
\newcommand*\Pair[1][{}]{\mathbin{\typecolor{\otimes}_{#1}}}
\newcommand*\PairUnord{\mathbin{\typecolor{\otimes}}}
\newcommand*\PairOrd{\mathbin{\typecolor{\odot}}}

% Kinds
\newcommand*\K{K}
\newcommand*\KSession[1]{\mathcal{S}^{#1}}
\newcommand*\KType[1]{\mathcal{T}^{#1}}

% Base Types
\newcommand*\B{\typecolor{B}}
\newcommand*\BUnit{\TypeKeyword{Unit}}
\newcommand*\BInt{\TypeKeyword{Int}}
\newcommand*\BBool{\TypeKeyword{Bool}}
\newcommand*\BString{\TypeKeyword{String}}

% Types
\newcommand*\T[1][{}]{\typecolor{T_{#1}}}
\newcommand*\TProd[3]{\typecolor{{#2}\otimes_{#1}{#3}}}
\newcommand*\TPair[3][{}]{\typecolor{{#2}\Pair[#1]{#3}}}
\newcommand*\TPairUnord[2]{\TProd\DirUnord{#1}{#2}} % \typecolor{{#1}\PairUnord{#2}}}
\newcommand*\TPairOrd[2]{\TProd\DirLeft{#1}{#2}} %{\typecolor{{#1}\PairOrd{#2}}}
\newcommand*\TSum[2]{\typecolor{{#1}+{#2}}}
\newcommand*\TVariant[2]{\typecolor{\langle{#1}\rangle_{#2}}}
\newcommand*\TArr[4]{\typecolor{{#3}\to^{#1}_{#2}{#4}}}
% \newcommand*\TArr[4]{{#3}\overset{{#1};{#2}}\to{#4}}
\newcommand*\TExists[3][{}]{\typecolor{\exists #2^{#1}. #3}}

% Unrestricted Types
\newcommand*\M{M}

% Priorities
\renewcommand*\O{\priocolor{o}}
\renewcommand*\P[1][{}]{\priocolor{p_{#1}}}
\newcommand*\Q[1][{}]{\priocolor{q_{#1}}}
\newcommand*\pr[1]{\metacolor{\operatorname{pr}(}#1\metacolor)}
\newcommand*\ptop{\priocolor{\top}}
\newcommand*\pbot{\priocolor{\bot}}
\newcommand*\plub{\mathbin{\priocolor{\sqcup}}}
\newcommand*\pglb{\mathbin{\priocolor{\sqcap}}}

% Session Types
\renewcommand*\S[1][{}]{\typecolor{S_{#1}}}
\newcommand*\R[1][{}]{\typecolor{R_{#1}}}
\newcommand*\SVar{\typecolor{X}}
\newcommand*\SSkip{\TypeKeyword{Skip}}
\newcommand*\SSeq[2]{\typecolor{{#1};{#2}}}
\newcommand*\SWait[1][{}]{\TypeKeyword{Wait}}
\newcommand*\STerm[1][{}]{\TypeKeyword{Close}} % Term}}
\newcommand*\SDual[1]{\typecolor{#1}^{\metacolor\bot}} % {\typecolor{\Dual\,({#1})}}
\newcommand*\STrans[2][{}]{\typecolor{\operatorname{\#}{#2}}}
\newcommand*\SSend[2][{}]{\typecolor{\operatorname{!}{\!#2}}}
\newcommand*\SRecv[2][{}]{\typecolor{\operatorname{?}{\!#2}}}
\newcommand*\SRecord[2]{\typecolor{\{{#1}\}_{#2}}}
\newcommand*\SVariant[2]{\typecolor{\langle{#1}\rangle_{#2}}}
\newcommand*\SMulti[3][{}]{\typecolor{\odot\SRecord{#2}{#3}}}
\newcommand*\SChoice[3][{}]{\typecolor{\oplus\SRecord{#2}{#3}}}
\newcommand*\SBranch[3][{}]{\typecolor{\&\SRecord{#2}{#3}}}
\newcommand*\SRec[2]{\typecolor{\mu{#1}.{#2}}}
\newcommand*\SRet{\TypeKeyword{Ret}}
\newcommand*\SAcq{\TypeKeyword{Acq}}

\newcommand*\SComp[1]{{#1}\mathbin{\metacolor{\fatsemi}}}

\newcommand\SSeen{\ctxcolor{\Theta}}
\newcommand\SeenEmpty{\ctxcolor{\emptyset}}
\newcommand\SeenSnoc[1]{\ctxcolor{;{}}({#1})}

\newcommand*\Sbot[1][{}]{\typecolor{S^{?}_{#1}}}
\newcommand\SNil{\typecolor{\bot}}
\newcommand\Ssub[2]{[{#2}/{#1}]}

\newcommand\SStrip[3][\SSeen]{{#1}\vdash{#2}\mathbin{\%}{#3}}
\DeclareMathOperator\Smerge{merge}

% Target
\newcommand*\TT{\targettypecolor{T}}
\newcommand*\ST{\targettypecolor{S}}
\newcommand*\STSend[2][{}]{\targettypecolor{\operatorname{!}^{\priocolor{#1}}{#2}.}}
\newcommand*\STRecv[2][{}]{\targettypecolor{\operatorname{?}^{\priocolor{#1}}{#2}.}}
\newcommand*\STWait[1][{}]{\targettypecolor{\Keyword{Wait}}^{\priocolor{#1}}}
\newcommand*\STTerm[1][{}]{\targettypecolor{\Keyword{Close}}^{\priocolor{#1}}} % Term}}
\newcommand*\TTPair[3][{}]{\targettypecolor{{#2}\otimes^{#1}{#3}}}
\newcommand*\STDual[1]{\targettypecolor{\Dual\,({#1})}}
\newcommand*\TBUnit{\TargetTypeKeyword{Unit}}

\newcommand*\TTVariant[2]{\targettypecolor{\langle{#1}\rangle_{#2}}}
\newcommand*\TTArr[3]{\targettypecolor{{#2}\to^{#1}{#3}}}

\newcommand*\TCtx{\targetctxcolor{\Gamma}}
\newcommand*\TCtxEmpty{\targetctxcolor{\emptyset}}

% Contexts
\newcommand*\Ctx[1][{}]{\ctxcolor{\Gamma_{#1}}}
\newcommand*\CtxEmpty{\ctxcolor{\emptyset}}
% \newcommand*\CtxEmpty{\ctxcolor{\cdot}}
\newcommand*\CtxVar[2]{\ctxcolor{{\termcolor{#1}}:{\typecolor{#2}}}}
\newcommand*\CtxExt[2]{\ctxcolor{{#1},{#2}}}
\newcommand*\CtxExtUnord[2]{\ctxcolor{{#1} \parallel {#2}}}
\newcommand*\CtxPatternRaw{\ctxcolor{\mathcal{G}}}
\newcommand*\CtxPattern[2][{}]{\ctxcolor{\mathcal{G}_{#1}[#2]}}
\newcommand*\CtxLeftPattern[1]{\ctxcolor{\mathcal{L}[#1]}}
\newcommand*\CtxExtInactive[4]{{#1},{#2}:^{#4}[{#3}]}
\newcommand*\CtxSeq[2]{\ctxcolor{{#1},{#2}}}
\newcommand*\CtxPar[2]{\ctxcolor{{#1}\parallel{#2}}}
\newcommand*\CtxSplit[4]{{#1}=^{#4}{#2}\fatsemi{#3}}
\newcommand*\CtxExtTarget[2]{{#1},{#2}}
% \newcommand*\CtxExtTarget[2]{{#1}\rhd{#2}}
\newcommand*\IsLeftPat{\operatorname{\textsf{LeftPat}}}
\newcommand*\CtxRestrict[2]{{#1}\downarrow{#2}}
\newcommand*\SubCtx{\preccurlyeq}
\newcommand*\CtxBind[2]{\ctxcolor{{#1}:{#2}}}

% Session Contexts
\newcommand*\SCtx{\Xi}
\newcommand*\SCtxEmpty{\cdot}
\newcommand*\SCtxExt[3]{{#1},{#2}:{#3}}

% binders
\newcommand*\BindGroup[1][{}]{\termcolor{B_{#1}}}
\newcommand*\Bind[1][{}]{\termcolor{b_{#1}}}
\newcommand*\BSeq[2]{\termcolor{#1; #2}}
\newcommand*\BPar[2]{\termcolor{#1\|#2}}
\newcommand*\BEmp{\termcolor{\cdot}}

% Expressions
\newcommand*\E[1][{}]{\termcolor{e_{#1}}}
\newcommand*\EVar[1][{}]{\termcolor{x_{#1}}}
\newcommand*\EVary{\termcolor{y}}
\newcommand*\EBorrow[1]{\termcolor{\&{#1}}}
\newcommand*\EAbs[3][{}]{\termcolor{\lambda_{#1}{#2}.{#3}}}
\newcommand*\EApp[2]{\termcolor{{#1}\,{#2}}}
\newcommand*\EAppD[3][{}]{\termcolor{({#2}\,{#3})_{#1}}} % with direction
\newcommand*\EInj[2]{\termcolor{\Keyword{inj$_{#1}$}{#2}}}
\newcommand*\EPair[3][{}]{\termcolor{{#2} \otimes_{#1} {#3}}}
\newcommand*\EPairOrd[2]{\EPair[\DirLeft]{#1}{#2}} % {\termcolor{{#1} \odot {#2}}}
\newcommand*\ECase[5]{\termcolor{\Keyword{case} {#1} \mathop{\{} \Keyword{inj$_1$} {#2} \Rightarrow{#3}; \Keyword{inj$_2$} {#4} \Rightarrow{#5} \mathop{\}}}}
\newcommand*\ELet[2]{\termcolor{\Keyword{let}\,{#1} \Keyword{=} {#2}\,\Keyword{in}\,}}
\newcommand*\ESeq[1]{\termcolor{{#1}\!\Keyword{;}}}
\newcommand*\ELetPair[4][{}]{\termcolor{\ELet{\EPair[#1]{#2}{#3}}{#4}}}
\newcommand*\ELetPairOrd[3]{\ELetPair[\DirLeft]{#1}{#2}{#3}} % {\termcolor{\ELet{\EPairOrd{#1}{#2}}{#3}}}
\newcommand*\EFork[1]{\termcolor{\CFork~{#1}}}
\newcommand*\EMatch[3]{\termcolor{\Keyword{match}\, {#1}\, \Keyword{with}\, \ERecord{#2}{#3}}}
\newcommand*\EMatchA[3]{\termcolor{\begin{array}[t]{@{}l@{}l}\Keyword{match}&\, {#1}\, \Keyword{with}\\&\ERecord{#2}{#3}\end{array}}}
\newcommand*\EPacked[2]{\termcolor{\langle#1,#2\rangle}}
\newcommand*\EPack[2]{\termcolor{\Keyword{pack}\,\langle#1,#2\rangle\,\Keyword{as}\,}}
\newcommand*\EOpen[3]{\termcolor{\Keyword{open}\,\langle#1,#2\rangle = #3\,\Keyword{in}\,}}
\newcommand*\ERecord[2]{\termcolor{\{{#1}\}_{#2}}}


\newcommand*\CRead{\targettermcolor{\mathit{read}}}
\newcommand*\CWrite{\targettermcolor{\mathit{write}}}
\newcommand*\CWrap{\targettermcolor{\mathit{wrap}}}

% Constants
\newcommand*\C{\termcolor{c}}
\newcommand*\CUnit{\TermKeyword{()}}
\newcommand*\CBase{\TermKeyword{b}}
\newcommand*\CFork{\TermKeyword{fork}}
\newcommand*\CSplit{\TermKeyword{lsplit}}
\newcommand*\CSSplit{\TermKeyword{rsplit}}
\newcommand*\CDrop{\TermKeyword{drop}}
\newcommand*\CAcq{\TermKeyword{acquire}}
\newcommand*\CNew{\TermKeyword{new}}
\newcommand*\CLink{\TermKeyword{link}}
\newcommand*\CWait{\TermKeyword{wait}}
\newcommand*\CTerm{\TermKeyword{close}} % term}}
\newcommand*\CSend{\TermKeyword{send}}
\newcommand*\CRecv{\TermKeyword{recv}} % recv}}
\newcommand*\CSelect[1]{\EApp{\TermKeyword{select}}{#1}}
\newcommand*\CBranch{\TermKeyword{branch}}
\newcommand*\COffer{\TermKeyword{offer}}


% target Expressions
\newcommand*\TE[1][{}]{\targettermcolor{e_{#1}}}
\newcommand*\TEVar[1][{}]{\targettermcolor{x_{#1}}}
\newcommand*\TEVary{\targettermcolor{y}}
\newcommand*\TEAbs[3][{}]{\targettermcolor{\lambda_{#1}{#2}.{#3}}}
\newcommand*\TEApp[2]{\targettermcolor{{#1}\,{#2}}}
\newcommand*\TEInj[2]{\targettermcolor{\Keyword{inj$_{#1}$}{#2}}}
\newcommand*\TEPair[3][{}]{\targettermcolor{{#2} \otimes_{#1} {#3}}}
\newcommand*\TECase[5]{\targettermcolor{\Keyword{case} {#1} \mathop{\{} \Keyword{inj$_1$} {#2} \Rightarrow{#3}; \Keyword{inj$_2$} {#4} \Rightarrow{#5} \mathop{\}}}}
\newcommand*\TELet[2]{\targettermcolor{\Keyword{let}\,{#1} \Keyword{=} {#2}\,\Keyword{in}\,}}
\newcommand*\TESeq[1]{\targettermcolor{{#1}\!\Keyword{;}}}
\newcommand*\TELetPair[4][{}]{\targettermcolor{\TELet{\TEPair[#1]{#2}{#3}}{#4}}}
\newcommand*\TEFork[1]{\targettermcolor{\TCFork~{#1}}}
\newcommand*\TEMatch[3]{\targettermcolor{\Keyword{match}\, {#1}\, \Keyword{with}\, \TERecord{#2}{#3}}}
\newcommand*\TEMatchA[3]{\targettermcolor{\begin{array}[t]{@{}l@{}l}\Keyword{match}&\, {#1}\, \Keyword{with}\\&\ERecord{#2}{#3}\end{array}}}
\newcommand*\TERecord[2]{\targettermcolor{\{{#1}\}_{#2}}}

% target constants
\newcommand*\TC{\targettermcolor{c}}
\newcommand*\TCUnit{\TargetTermKeyword{()}}
\newcommand*\TCBase{\TargetTermKeyword{b}}
\newcommand*\TCFork{\TargetTermKeyword{spawn}}
\newcommand*\TCSplit{\TargetTermKeyword{split}}
\newcommand*\TCNew{\TargetTermKeyword{new}}
\newcommand*\TCLink{\TargetTermKeyword{link}}
\newcommand*\TCWait{\TargetTermKeyword{wait}}
\newcommand*\TCTerm{\TargetTermKeyword{close}} % term}}
\newcommand*\TCSend{\TargetTermKeyword{send}}
\newcommand*\TCRecv{\TargetTermKeyword{recv}} % recv}}
\newcommand*\TCSelect[1]{\EApp{\TargetTermKeyword{select}}{#1}}
\newcommand*\TCBranch{\TargetTermKeyword{branch}}
\newcommand*\TCOffer{\TargetTermKeyword{offer}}
\newcommand*\TCDrop{\TargetTermKeyword{drop}}

% Values
\newcommand*\V{\termcolor{v}}
\newcommand*\U{\termcolor{u}}

% Evaluation Contexts
\newcommand*\EC{\termcolor{E}}
\newcommand*\ECEmpty{\termcolor{\Box}}

% Flagged Contexts
\newcommand*\FC[1][{}]{\termcolor{\mathcal{F}_{#1}}}
\newcommand*\FCapp[2][{}]{\termcolor{\mathcal{F}_{#1}[#2]}}

% Configuration Contexts
\newcommand*\CC[1][{}]{\termcolor{\mathcal{G}_{#1}}}
\newcommand*\CCapp[2][{}]{\termcolor{\mathcal{G}_{#1}[#2]}}

% Processes
\newcommand*\Proc[1][{}]{\termcolor{{P}_{#1}}}
\newcommand*\PExp[2][{}]{\termcolor{{}^{#1}\langle{#2}\rangle}}
\newcommand*\PNew[3]{\termcolor{(\nu [{#1}][{#2}]){#3}}}
\newcommand*\PPar[2]{\termcolor{{#1} \,\|\, {#2}}}

% Variable Sets ("Dependencies")
\newcommand*\D{\Delta}
\newcommand*\DVar{\delta}
\newcommand*\DSingle[1]{\{{#1}\}}
\newcommand*\DEmpty{\emptyset}
\newcommand*\DJoin{\cup}

% Effects
\newcommand*\Eff[1][{}]{\typecolor{\epsilon_{#1}}}
\newcommand*\EffPure{\typecolor{\textsc{p}}}
\newcommand*\EffImpure{\typecolor{\textsc{i}}}
% \newcommand*\EffPure{\typecolor{\mathbf{0}}}
% \newcommand*\EffImpure{\typecolor{\mathbf{1}}}

% Effect Order
\newcommand*\EffLeq{\sqsubseteq}
\newcommand*\EffJoin{\metacolor\sqcup} 

% Flags
\newcommand*\Flag{\termcolor{\phi}}
\newcommand*\FlagMain{\termcolor{\bullet}}
\newcommand*\FlagAux{\termcolor{\circ}}

% Graphs
\newcommand*\Graph{G}
\newcommand*\Node{n}

% Session Type Equivalence
\newcommand*\SEq{\approx}

% Free Variables
\newcommand*\Fv{\operatorname{\mathsf{fv}}}

% Domain of a type context
\newcommand*\CtxDom{\operatorname{\mathsf{dom}}}

% Dual Session Type
\newcommand*\Dual{{\operatorname{\mathsf{dual}}}}

% Ordered Types
\newcommand*\Ord{\operatorname{\mathsf{ord}}}

% Unrestricted Types
\newcommand*\Unr{\operatorname{\mathsf{unr}}}

% Terminated Session Types
\newcommand*\Bounded{\operatorname{\mathsf{bounded}}}
\newcommand*\Mobile{\operatorname{\mathsf{mobile}}}
\newcommand*\Skips{\operatorname{\mathsf{skips}}}

% Session Type Formation
\newcommand*\HasSKind[4]{{#1}\mid{#2}\vdash{#3}:{#4}}

% Regular Type Formation
\newcommand*\HasTKind[3]{{#1}\vdash{#2}:{#3}}

% Expression Typing
\newcommand*\HasType[5][{}]{{\ctxcolor{#2}}\vdash{\termcolor{#3}}:{\typecolor{#4}}\mid{\typecolor{#5}}}

% Constant Typing
\newcommand*\HasCType[2]{{#1}:{#2}}

% Process Typing
\newcommand*\ProcHasType[2]{\ctxcolor{#1} \vdash {\termcolor{#2}} }

% Type Context Formation
\newcommand*\CtxWf[1]{\vdash{#1}}

% Multiplicity Order
\newcommand*\MulLeq{\leq}

% Subtyping for Sessions
\newcommand*\IsSSubtypeOf{\operatorname{<:^s}}

% Subtyping for Types
\newcommand*\IsTSubtypeOf{\operatorname{<:^t}}

% Algorithmic Expression Typing
\newcommand*\TypeInfer[5]{{#1}\vdash{#2}\uparrow{#3}\mid{#4}\mid{#5}}
\newcommand*\TypeCheck[5]{{#1}\vdash{#2}\downarrow{#3}\mid{#4}\mid{#5}}

\newcommand*\TypeInferX[5][]{{#2}\vdash{#3}\uparrow^{#1}{#4}\dashv{#5}}
\newcommand*\TypeCheckX[5][]{{#2}\vdash{#3}\downarrow^{#1}{#4}\dashv{#5}}

\newcommand*\CtxDiff[2]{{#1} \operatorname{\backslash} {#2}}
\newcommand*\IfThenElse[3]{\Keyword{if}{#1}\Keyword{then}{#2}\Keyword{else}{#3}}

% Reduction 
\newcommand{\Reduces}{\rightarrow}
\newcommand{\IsRed}[2]{{#1}\;{\reduces}\;{#2}}

% Operators 
\newcommand{\Subs}[3]{\termcolor{{#1}[{#2}/{#3}]}}

% Congruence
\newcommand{\Congruent}{\equiv}

% Translations
\newcommand*\TA[2]{\metacolor{\llbracket #1;#2 \rrbracket_{\Gamma}}}
% \newcommand*\TA[2]{\metacolor{\llbracket #2 \rrbracket_{\Gamma} ({#1})}}
% {\metacolor{\mathcal{T}_{\Ctx}( #1 , #2 )}}
\newcommand*\TTA[1]{\metacolor{\llbracket #1 \rrbracket_{\textsc t}}} % {\metacolor{\mathcal{T}_{\T} (#1)}}
\newcommand*\TS[2]{\metacolor{\llbracket #1, #2 \rrbracket_{\textsc s}}} % {\metacolor{\mathcal{T}_{\S} (#1, #2)}}
\newcommand*\TSA[1]{\metacolor{\llbracket #1 \rrbracket_{\textsc s}}}% {\metacolor{\mathcal{T}_{\S} (#1)}}
\newcommand*\TCA[1]{\metacolor{\llbracket #1\rrbracket_{\textsc c}}}
%{\metacolor{\mathcal{T}_{\C} (#1)}}
\newcommand*\TEA[1]{\metacolor{\llbracket #1 \rrbracket_{\textsc e}}}% {\metacolor{\mathcal{T}_{\E} (#1)}}

\newcommand*\Cell[2][{}]{\targettypecolor{\Keyword{Cell}^{\priocolor{#1}}} (#2)}

% Predicates on types
\newcommand*\IsBorrow[1]{\Keyword{Borrow}~#1}
\newcommand*\IsOwning[1]{\Keyword{Owning}~#1}

% General math
\newcommand*\Nat{\mathbf{N}}

% Target types
\newcommand*\HasTargetType[4][{}]{#2 \vdash^{#1} {#3}:{#4} }

% Notes
\newcommand{\note}[2]{{\color{red}[#1:} #2{\color{red}]}}
\newcommand{\vv}[1]{\note{vv}{#1}}
\newcommand{\pt}[1]{\note{pt}{#1}}
\newcommand{\hs}[1]{\note{hs}{#1}}

%%% Local Variables:
%%% mode: latex
%%% TeX-master: "main"
%%% End:

% Unrestricted Types

\newcommand*\RuleUBase{
  \inferrule[U-Base]{}{
    \Unr\B
  }
}

\newcommand*\RuleUProd{
  \inferrule[U-Prod]{
    \Unr{\T_1} \\
    \Unr{\T_2}
  } {
    \Unr{(\TPair{\T_1}{\T_2})}
  }
}

% \newcommand*\RuleUSum{
%   \inferrule[U-Sum]{
%     \Unr{\T_1} \\
%     \Unr{\T_2}
%   } {
%     \Unr{(\TSum{\T_1}{\T_2})}
%   }
% }

\newcommand*\RuleUArr{
  \inferrule[U-Arr]{}{
    \Unr{(\TArr\DirNone\Eff{\T_1}{\T_2})}
  }
}

\newcommand*\RuleUSkip{
  \inferrule[U-Skip]{}{
    \Unr{\SSkip}
  }
}

\newcommand*\RuleUSeq{
  \inferrule[U-Seq]{
    \Unr{\S_1} \\
    \Unr{\S_2}
  } {
    \Unr{(\SSeq{\S_1}{\S_2})}
  }
}

\newcommand*\RuleUVar{
  \inferrule[U-Var]{}{
    \Unr{\SVar}
  }
}

\newcommand*\RuleURec{
  \inferrule[U-Rec]{
    \Unr{\S}
  } {
    \Unr{(\SRec\SVar\S)}
  }
}

%%% Local Variables:
%%% mode: latex
%%% TeX-master: "../main"
%%% End:

% Ordered Types

\newcommand*\RuleOProdOne{
  \inferrule[O-Prod1]{
    \Ord{\T_1} 
  } {
    \Ord{(\TPair{\T_1}{\T_2})}
  }
}

\newcommand*\RuleOProdTwo{
  \inferrule[O-Prod2]{
    \Ord{\T_2} 
  } {
    \Ord{(\TPair{\T_1}{\T_2})}
  }
}

\newcommand*\RuleOArr{
  \inferrule[O-Arr]{}{
    \Ord{(\TArr\Dir\Eff{\T_1}{\T_2})}
  }
}

% Mobile and Bounded Session Types

\newcommand*\RuleTeAcq{
  \inferrule[Te-Acq]{\Bounded\S}{\Mobile{(\SSeq\SAcq\S)}}
}

\newcommand*\RuleTeTerm{
  \inferrule[Te-Term]{}{
    \Bounded\STerm
  }
}

\newcommand*\RuleTeRet{
  \inferrule[Te-Rel]{}{
    \Bounded\SRet
  }
}

\newcommand*\RuleTeSeqOne{
  \inferrule[Te-Seq1]{
    \Bounded{\S_1}
  } {
    \Bounded{(\SSeq{\S_1}{\S_2})}
  }
}

\newcommand*\RuleTeSeqTwo{
  \inferrule[Te-Seq2]{
    \Bounded{\S_2}
  } {
    \Bounded{(\SSeq{\S_1}{\S_2})}
  }
}

\newcommand*\RuleTeBranch{
  \inferrule[Te-Branch]{
    \forall \ell.\ \Bounded{\S_\ell}
  } {
    \Bounded{\SBranch{\ell:\S_\ell}{\ell\in L}}
  }
}

\newcommand*\RuleTeChoice{
  \inferrule[Te-Choice]{
    \forall \ell.\ \Bounded{\S_\ell}
  } {
    \Bounded{\SChoice{\ell:\S_\ell}{\ell\in L}}
  }
}

\newcommand*\RuleTeVar{
  \inferrule[Te-Var]{}{
  \Bounded\SVar 
  }
}

\newcommand*\RuleTeRec{
  \inferrule[Te-Rec]{
  \Bounded\S
  }{
  \Bounded{\SRec\SVar\S}
  }
}

\newcommand*\RuleTeBase{
  \inferrule[Te-Base]{}{
    \Mobile{\BUnit}
  }
}

\newcommand*\RuleTeProd{
  \inferrule[Te-Prod]{
    \Mobile{\T[1]} \\
    \Mobile{\T[2]}
  }{
    \Mobile{(\TPair{\T[1]}{\T[2]})}
  }
}

\newcommand*\RuleTeVariant{
  \inferrule[Te-Var]{
    \forall\ell. \Mobile \T[\ell]
  }{
    \Mobile{(\TVariant{\T[\ell]}{\ell\in L})}}
}

\newcommand*\RuleTeArr{
  \inferrule[Te-Arr]{}{
    \Mobile{(\TArr{\MobMobile}{\Dir,\Eff}{\T_1}{\T_2})}
  }
}

\newcommand*\RuleTeCtxEmpty{
  \inferrule[Te-Ctx-Empty]{}{
    \Mobile \CtxEmpty
  }
}

\newcommand*\RuleTeCtxSeq{
  \inferrule[Te-Ctx-Seq]{
    \Mobile \Ctx[1] \\
    \Mobile \Ctx[2]
  }{
    \Mobile (\CtxSeq{\Ctx[1]}{\Ctx[2]})
  }
}

\newcommand*\RuleTeCtxPar{
  \inferrule[Te-Ctx-Par]{
    \Mobile \Ctx[1] \\
    \Mobile \Ctx[2]
  }{
    \Mobile (\CtxPar{\Ctx[1]}{\Ctx[2]})
  }
}

\newcommand*\RuleTeCtxVar{
  \inferrule[Te-Ctx-Var]{
    \Mobile \T
  }{
    \Mobile (\CtxVar \EVar \T)
  }
}

%%% Local Variables:
%%% mode: latex
%%% TeX-master: "../main"
%%% End:

% % Session Type Formation

\newcommand*\RuleSFWait{
  \inferrule[SF-Wait]{}{
    \HasSKind{\Ctx}{\SCtx}{\SWait}{\KSession\MulLin}
  }
}

\newcommand*\RuleSFSend{
  \inferrule[SF-Send]{
    \HasTKind{\Ctx}{\M}{\KType\MulUnr}
  }{
    \HasSKind{\Ctx}{\SCtx}{\SSend\M}{\KSession\MulLin}
  }
}

\newcommand*\RuleSFBranch{
  \inferrule[SF-Branch]{
    \forall i.\ \HasSKind{\Ctx}{\SCtx}{\S_i}{\KSession{\Mul_i}}
  }{
    \HasSKind{\Ctx}{\SCtx}{\SBranch{i:\S_i}}{\KSession\MulLin}
  }
}

\newcommand*\RuleSFTerm{
  \inferrule[SF-Term]{}{
    \HasSKind{\Ctx}{\SCtx}{\STerm}{\KSession\MulLin}
  }
}

\newcommand*\RuleSFRecv{
  \inferrule[SF-Recv]{
    \HasTKind{\Ctx}{\M}{\KType\MulUnr}
  }{
    \HasSKind{\Ctx}{\SCtx}{\SRecv\M}{\KSession\MulLin}
  }
}

\newcommand*\RuleSFChoice{
  \inferrule[SF-Choice]{
    \forall i.\ \HasSKind{\Ctx}{\SCtx}{\S_i}{\KSession{\Mul_i}}
  }{
    \HasSKind{\Ctx}{\SCtx}{\SChoice{i:\S_i}}{\KSession\MulLin}
  }
}

\newcommand*\RuleSFSkip{
  \inferrule[SF-Skip]{}{
    \HasSKind{\Ctx}{\SCtx}{\SSkip}{\KSession\MulUnr}
  }
}

\newcommand*\RuleSFSeq{
  \inferrule[SF-Seq]{
    \HasSKind{\Ctx}{\SCtx}{\S_1}{\KSession{\Mul_1}} \\
    \HasSKind{\Ctx}{\SCtx}{\S_2}{\KSession{\Mul_2}} \\
    \Mul_1 \MulLeq \Mul \\
    \Mul_2 \MulLeq \Mul
  }{
    \HasSKind{\Ctx}{\SCtx}{(\SSeq{\S_1}{\S_2})}{\KSession\Mul}
  }
}

\newcommand*\RuleSFRec{
  \inferrule[SF-Rec]{
    \HasSKind{\Ctx}{\SCtxExt\SCtx\SVar{\KSession\Mul}}{\S}{\KSession\Mul}
  }{
    \HasSKind{\Ctx}{\SCtx}{(\SRec\SVar\S)}{\KSession\Mul}
  }
}

\newcommand*\RuleSFVar{
  \inferrule[SF-Var]{
    (\SVar : \KSession\Mul) \in \SCtx
  }{
    \HasSKind{\Ctx}{\SCtx}{\SVar}{\KSession\Mul}
  }
}


% % Type Formation

\newcommand*\RuleTFBase{
  \inferrule[TF-Base]{}{
    \HasTKind{\Ctx}{\B}{\KType\MulUnr}
  }
}

\newcommand*\RuleTFArr{
  \inferrule[TF-Arr]{
    \HasTKind{\Ctx}{\T_1}{\KType{\Mul_1}} \\
    \HasTKind{\Ctx}{\T_2}{\KType{\Mul_2}}
  }{
    \HasTKind{\Ctx}{(\TArr\Mul{\T_1}{\T_2})}{\KType\Mul}
  }
}

\newcommand*\RuleTFProd{
  \inferrule[TF-Prod]{
    \HasTKind{\Ctx}{\T_1}{\KType{\Mul_1}} \\
    \HasTKind{\Ctx}{\T_2}{\KType{\Mul_2}} \\
    \Mul_1 \MulLeq \Mul \\
    \Mul_2 \MulLeq \Mul
  }{
    \HasTKind{\Ctx}{(\TProd\Dir{\T_1}{\T_2})}{\KType\Mul}
  }
}


% Context Formation

\newcommand*\RuleCFEmpty{
  \inferrule[CF-Empty]{}{
    \CtxWf{\CtxEmpty}
  }
}

\newcommand*\RuleCFExt{
  \inferrule[CF-Ext]{
    \CtxWf{\Ctx} \\
    \EVar \not\in \CtxDom\Ctx \\
    \D \subseteq \CtxDom\Ctx
  }{
    \CtxWf{\CtxExt\Ctx\EVar\T\D}
  }
}

\newcommand*\RuleCFExtInactive{
  \inferrule[CF-ExtInactive]{
    \CtxWf{\Ctx} \\
    \EVar \not\in \CtxDom\Ctx \\
    \D \subseteq \CtxDom\Ctx
  }{
    \CtxWf{\CtxExtInactive\Ctx\EVar\T\D}
  }
}

\newcommand*\RuleCFExtMeta{
  \inferrule[CF-ExtMeta]{
    \CtxWf{\Ctx} \\
    \DVar \not\in \CtxDom\Ctx
  }{
    \CtxWf{\CtxExtD\Ctx\DVar}
  }
}


% Session Subtyping

\newcommand*\RuleSSTBranch{
  \inferrule[SST-Branch]{}{
    \SBranch{i:\S_i} \IsSSubtypeOf \SBranch{i:\S_i, j:\S_j}
  }
}

\newcommand*\RuleSSTChoice{
  \inferrule[SST-Choice]{}{
    \SChoice{i:\S_i, j:\S_j} \IsSSubtypeOf \SChoice{i:\S_i}
  }
}

\newcommand*\RuleSSTSeq{
  \inferrule[SST-Seq]{
    \S_1 \IsSSubtypeOf \S_1' \\
    \S_2 \IsSSubtypeOf \S_2'
  }{
    \SSeq{\S_1}{\S_2} \IsSSubtypeOf \SSeq{\S_1'}{\S_2'}
  }
}

\newcommand*\RuleSSTTerm{
  \inferrule[SST-Term]{}{
    \STerm \IsSSubtypeOf \STerm
  }
}

\newcommand*\RuleSSTWait{
  \inferrule[SST-Wait]{}{
    \SWait \IsSSubtypeOf \SWait
  }
}

\newcommand*\RuleSSTSend{
  \inferrule[SST-Send]{}{
    \SSend\M \IsSSubtypeOf \SSend\M
  }
}

\newcommand*\RuleSSTRecv{
  \inferrule[SST-Recv]{}{
    \SRecv\M \IsSSubtypeOf \SRecv\M
  }
}

\newcommand*\RuleSSTRecL{
  \inferrule[SST-RecL]{
    \S_1[\SVar := \SRec\SVar{\S_1}] \IsSSubtypeOf \S_2
  }{
    \SRec\SVar{\S_1} \IsSSubtypeOf \S_2
  }
}

\newcommand*\RuleSSTRecR{
  \inferrule[SST-RecR]{
    \S_1 \IsSSubtypeOf \S_2[\SVar := \SRec\SVar{\S_2}]
  }{
    \S_1 \IsSSubtypeOf \SRec\SVar{\S_2}
  }
}


%%% Local Variables:
%%% mode: latex
%%% TeX-master: "../main"
%%% End:

% Type Subtyping

\newcommand*\RuleTSTSession{
  \inferrule[TST-Session]{
    \S_1 \IsSSubtypeOf \S_2
  }{
    \S_1 \IsTSubtypeOf \S_2
  }
}

\newcommand*\RuleTSTArr{
  \inferrule[TST-Arr]{
    \T_1' \IsTSubtypeOf \T_1 \\
    \T_2 \IsTSubtypeOf \T_2' \\
    \Mul \MulLeq \Mul' \\
    \Eff \EffLeq \Eff'
  }{
    \TArr{\Mul}{\Eff}{\T_1}{\T_2} \IsTSubtypeOf \TArr{\Mul'}{\Eff'}{\T_1'}{\T_2'}
  }
}

% Type Context Splitting


\newcommand*\RuleCSEmpty{
  \inferrule[CS-Empty]{}{
    \CtxSplit\CtxEmpty\CtxEmpty\CtxEmpty\D
  }
}

\newcommand*\RuleCSSplit{
  \inferrule[CS-Split]{
    \CtxSplit
      {\Ctx}
      {\Ctx_1}
      {\Ctx_2}
      {\D\cup\D'}
    \\
    \EVar \not\in \D
  }{
    \CtxSplit
      {\CtxExt\Ctx\EVar{(\SSeq{\S_1}{\S_2})}{\D'}}
      {\CtxExt{\Ctx_1}\EVar{\S_1}{\D'}}
      {\CtxExt{\Ctx_2}\EVar{\S_2}{\D'}}
      {\D}
  }
}

\newcommand*\RuleCSLeft{
  \inferrule[CS-Left]{
    \CtxSplit
      {\Ctx}
      {\Ctx_1}
      {\Ctx_2}
      {\D}
    \\
    \EVar \not\in \D
  }{
    \CtxSplit
      {\CtxExt\Ctx\EVar\T{\D'}}
      {\CtxExt{\Ctx_1}\EVar\T{\D'}}
      %{\CtxExtInactive{\Ctx_2}\EVar\T{\D'}}
      {\Ctx_2}
      {\D}
  }
}

\newcommand*\RuleCSRight{
  \inferrule[CS-Right]{
    \CtxSplit
      {\Ctx}
      {\Ctx_1}
      {\Ctx_2}
      {\D\cup\D'}
  }{
    \CtxSplit
      {\CtxExt\Ctx\EVar\T{\D'}}
      {\CtxExtInactive{\Ctx_1}\EVar\T{\D'}}
      {\CtxExt{\Ctx_2}\EVar\T{\D'}}
      {\D}
  }
}

\newcommand*\RuleCSBoth{
  \inferrule[CS-Both]{
    \CtxSplit
      {\Ctx}
      {\Ctx_1}
      {\Ctx_2}
      {\D}
    \\
    \Unr \T
  }{
    \CtxSplit
      {\CtxExt\Ctx\EVar\T{\emptyset}}
      {\CtxExt{\Ctx_1}\EVar\T{\emptyset}}
      {\CtxExt{\Ctx_2}\EVar\T{\emptyset}}
      {\D}
  }
}

% Constant Typing

\newcommand*\RuleCTLink{
  \inferrule[CT-Link]{}{
    \HasCType{\CLink_{\S}}{\TArr{\MobMobile}{\DirUnord,\EffPure} {\TPairUnord{\S}{\Dual\S}}\BUnit}
  }
}

\newcommand*\RuleCTUnit{
  \inferrule[CT-Unit]{}{
    \HasCType\CUnit\BUnit
  }
}

\newcommand*\RuleCTFork{
  \inferrule[CT-Fork]{}{
    \HasCType\CFork{
      \TArr{\MobMobile}{\DirUnord,\EffPure}{
        (\TArr{\MobMobile}{\DirUnord,\EffImpure}\BUnit\BUnit)
      }{\BUnit}
    }
  }
}

\newcommand*\RuleCTNew{
  \inferrule[CT-New]{}{
    \HasCType{\CNew_{\S}}{\TArr{\MobMobile}{\DirUnord,\EffPure}{\BUnit}{\TPairUnord{(\SSeq{\SAcq}{\S})}{(\SSeq{\SAcq}{\Dual~\S})}}}
  }
}

\newcommand*\RuleCTDrop{
  \inferrule[CT-Drop]{}{
    \HasCType
    {\CDrop}{\TArr{\MobMobile}{\DirUnord,\EffPure}{\SRet}{\BUnit}}
  }
}

\newcommand*\RuleCTAcquire{
  \inferrule[CT-Acquire]{}{
    \HasCType
    {\CAcq}{\TArr{\MobMobile}{\DirUnord,\EffPure}{\SSeq\SAcq\S}{\S}}
  }
}

\newcommand*\RuleCTSplit{
  \inferrule[CT-LSplit]{}{
    \HasCType{\CSplit_{\S[1],\S[2]}}{\TArr{\MobMobile}{\DirUnord,\EffPure}{\SSeq {\S_1}{\S_2}}{\TPairOrd{\S_1}{\S_2}}}
  }
}

\newcommand*\RuleCTSSplit{
  \inferrule[CT-RSplit]{}{
    \HasCType
    {\CSSplit_{\S[1],\S[2]}}
    {\TArr{\MobMobile}{\DirUnord,
      \EffPure}
      {\SSeq {\S_1}{\S_2}}
      {\TPairUnord{(\SSeq{\S_1}{\SRet})}{(\SSeq{\SAcq}{\S_2})}}}
  }
}

\newcommand*\RuleCTWait{
  \inferrule[CT-Wait]{}{
    \HasCType\CWait{\TArr{\MobMobile}{\DirUnord,\EffImpure}{\SWait[\O]}\BUnit}
  }
}

\newcommand*\RuleCTTerm{
  \inferrule[CT-Term]{}{
    \HasCType\CTerm{\TArr{\MobMobile}{\DirUnord,\EffImpure}{\STerm[\O]}\BUnit}
  }
}

\newcommand*\RuleCTSend{
  \inferrule[CT-Send]{
    \Mobile \T
  }{
    \HasCType\CSend{\TArr{\MobMobile}{\DirUnord,\EffImpure}{{\T}\PairUnord{\SSend[\O]\T}}\BUnit}
  }
}

\newcommand*\RuleCTRecv{
  \inferrule[CT-Recv]{
    \Mobile \T
  }{
    \HasCType\CRecv{\TArr{\MobMobile}{\DirUnord,\EffImpure}{\SRecv[\O]\T}\T}
  }
}

\newcommand*\RuleCTSelect{
  \inferrule[CT-Select]{}{
    \HasCType{\CSelect{k}}{\TArr{\MobMobile}{\SChoice{\ell:\S_\ell}{\ell\in L}}\S_k}
  }
  \\
  \inferrule[CT-Select']{}{
    \HasCType{\CSelect{k}}{\TArr{\MobMobile}{\SChoice{k:\SSkip}{}}\BUnit}
  }
}

\newcommand*\RuleCTBranch{
  \inferrule[CT-Branch]{}{
    \HasCType\CBranch{\TArr{\MobMobile}{\SBranch{\ell:\S_\ell}{\ell\in L}}{\Sigma_k \S_k}}
  }
}

%%%%%%%%%%%%%%%%%%%%%%%%%%%%%%%%%%%%%%%%%%%%%%%%%%%%%%%%%%%%


\newcommand*\RuleCTNewAlt{
  \inferrule[CT-New]{}{
    \HasCType{\CNew_{\S}}{\TArr{\MobMobile}{\DirUnord,\EffPure}{\BUnit}{\TPairUnord{\S^\blacksquare}{{\SDual\S}^\blacksquare}}}
  }
}

\newcommand*\RuleCTSplitAlt{
  \inferrule[CT-Split]{}{
    \HasCType{\CSplit_{\S[1],\S[2]}}{\TArr{\MobMobile}{\DirUnord,\EffPure}{{(\SSeq {\S[1]}{\S[2]})}^{\K}}{\TPairOrd{\S[1]^{\S[2]\fatsemi\K}}{\S[2]^{\K}}}}
  }
}

%%% Local Variables:
%%% mode: latex
%%% TeX-master: "../main"
%%% End:

% Expression Typing

\newcommand*\RuleTConst{
  \inferrule[T-Const]{
    \HasCType
      \C
      \T
  }{
    \HasType[\pbot]
      \CtxEmpty
      \C
      \T
      \EffPure
  }
}

\newcommand*\RuleTVar{
  \inferrule[T-Var]{}{
    \HasType[\pbot]
      {\CtxVar \EVar\T}
      \EVar
      \T
      \EffPure
  }
}

\newcommand*\RuleTFork{
  \inferrule[T-Fork]{
    \Bounded \Ctx
    \\
    \HasType
      {\Ctx}
      {\E}
      {\TArr{\MobMobile}{\Eff} {\BUnit} {\BUnit}}
      \Eff[1]
  }{
    \HasType
      \Ctx
      {\EFork \E}
      {\BUnit}
      \EffPure
  }
}

\newcommand*\RuleTAbsUnr{
  \inferrule[T-AbsUnr]{
    \Unr\Ctx \\
    \HasType[\Q]
      {\CtxExt\Ctx{\CtxVar\EVar{\T[1]}}}
      {\E}
      {\T[2]}
      \Eff
  }{
    \HasType[\pbot]
      \Ctx
      {(\EAbs\EVar\E)}
      {(\TArr{\MobMobile}{\DirNone,\Eff}{\T[1]}{\T[2]})}
      \EffPure
  }
}

\newcommand*\RuleTAbsLin{
  \inferrule[T-AbsLin]{
    \HasType[\Q]
      {\CtxExtUnord\Ctx{\CtxVar\EVar{\T[1]}}}
      {\E}
      {\T[2]}
      \Eff \\
      \Mob (\Ctx)
  }{
    \HasType[\pbot]
      \Ctx
      {(\EAbs\EVar\E)}
      {(\TArr{\Mob}{\DirUnord,\Eff}{\T[1]}{\T[2]})}
      \EffPure
  }
}

\newcommand*\RuleTAbsLeft{
  \inferrule[T-AbsLeft]{
    \HasType[\Q]
      {\CtxExt{\CtxVar{\EVar}{\T[1]}}\Ctx}
      {\E}
      {\T[2]}
      \Eff \\
      \Mob (\Ctx)
  }{
    \HasType[\pbot]
      \Ctx
      {(\EAbs\EVar\E)}
      {(\TArr{\Mob}{\DirLeft,\Eff}{\T[1]}{\T[2]})}
      \EffPure
  }
}

\newcommand*\RuleTAbsRight{
  \inferrule[T-AbsRight]{
    \HasType[\Q]
      {\CtxExt\Ctx{\CtxVar{\EVar}{\T[1]}}}
      {\E}
      {\T[2]}
      \Eff \\
      \Mob (\Ctx)
  }{
    \HasType[\pbot]
      \Ctx
      {(\EAbs\EVar\E)}
      {(\TArr{\Mob}{\DirRight,\Eff}{\T[1]}{\T[2]})}
      \EffPure
  }
}

\newcommand*\RuleTAppUnr{
  \inferrule[T-AppUnr]{
    \HasType[\Q[1]]
      {\Ctx_1}
      {\E[1]}
      {(\TArr{\Mob}{\DirNone,{\Eff}[3]}{\T[1]}{\T[2]})}
      {\Eff[1]} \\
    \HasType[\Q[2]]
      {\Ctx_2}
      {\E[2]}
      {\T[1]}
      {\Eff[2]}
  }{
    \HasType[\P[1] \plub \P[2] \plub \Q[3]]
      {\CtxSeq {\Ctx_1}{\Ctx_2}}
      {(\EApp{\E[1]}{\E[2]})}
      {\T[2]}
      {(\Eff[1] \EffJoin \Eff[2] \EffJoin \Eff[3])}
  }
}

\newcommand*\RuleTAppLin{
  \inferrule[T-AppLin]{
    \HasType[{\Q[1]}]
      {\Ctx_1}
      {\E[1]}
      {(\TArr{\Mob}{\DirUnord,{\Eff[3]}}{\T[1]}{\T[2]})}
      {\Eff[1]} \\
    \HasType[{\Q[2]}]
      {\Ctx_2}
      {\E[2]}
      {\T[1]}
      {\Eff[2]}
  }{
    \HasType[{\P[1] \plub \P[2] \plub \Q[3]}]
      {\CtxPar {\Ctx_1}{\Ctx_2}}
      {(\EApp{\E[1]}{\E[2]})}
      {\T[2]}
      {(\Eff[1] \EffJoin \Eff[2] \EffJoin \Eff[3])}
  }
}

\newcommand*\RuleTAppLeft{
  \inferrule[T-AppLeft]{
    % \Unr{\Ctx_1} \vee \Eff[1] = \EffPure
    % \\
    \HasType[{\Q[1]}]
      {\Ctx_1}
      {\E[1]}
      {(\TArr{\Mob}{\DirLeft,{\Eff[3]}}{\T[1]}{\T[2]})}
      {\EffPure} \\
    \HasType[{\Q[2]}]
      {\Ctx_2}
      {\E[2]}
      {\T[1]}
      {\Eff[2]}
  }{
    \HasType[{\P[1] \plub \P[2] \plub \Q[3]}]
      {\CtxSeq {\Ctx_2}{\Ctx_1}}
      {(\EApp{\E[1]}{\E[2]})}
      {\T[2]}
      {(\Eff[2] \EffJoin \Eff[3])}
  }
}

\newcommand*\RuleTAppRight{
  \inferrule[T-AppRight]{
    % \Unr{\Ctx_2} \vee \Eff[2] = \EffPure
    % \\
    \HasType[{\Q[1]}]
      {\Ctx_1}
      {\E[1]}
      {(\TArr{\Mob}{\DirRight,{\Eff[3]}}{\T[1]}{\T[2]})}
      {\Eff[1]} \\
    \HasType[{\Q[2]}]
      {\Ctx_2}
      {\E[2]}
      {\T[1]}
      {\EffPure}
  }{
    \HasType[{\P[1] \plub \P[2] \plub \Q[3]}]
      {\CtxSeq {\Ctx_1}{\Ctx_2}}
      {(\EApp{\E[1]}{\E[2]})}
      {\T[2]}
      {(\Eff[1] \EffJoin \Eff[3])}
  }
}

\newcommand*\RuleTPairUnord{
  \inferrule[T-PairUnord]{
    \HasType[{\Q[1]}]
      {\Ctx_1}
      {\E[1]}
      {\T[1]}
      {\Eff[1]} \\
    \HasType[{\Q[2]}]
      {\Ctx_2}
      {\E[2]}
      {\T[2]}
      {\Eff[2]}
  }{
    \HasType[{\Q[1] \plub \Q[2]}]
      {\CtxPar {\Ctx_1}{\Ctx_2}}
      {\EPair{\E[1]}{\E[2]}}
      {\TPairUnord{\T[1]}{\T[2]}}
      {(\Eff[1] \EffJoin \Eff[2])}
  }
}

\newcommand*\RuleTPairOrd{
  \inferrule[T-PairOrd]{
    % \Unr{\Ctx_1} \vee \Unr {\T[1]} \vee \Eff[2] = \EffPure \\
    \HasType[{\Q[1]}]
      {\Ctx_1}
      {\E[1]}
      {\T[1]}
      {\Eff[1]} \\
    \HasType[{\Q[2]}]
      {\Ctx_2}
      {\E[2]}
      {\T[2]}
      {\EffPure}
  }{
    \HasType[{\Q[1] \plub \Q[2]}]
      {\CtxSeq {\Ctx_1}{\Ctx_2}}
      {\EPair{\E[1]}{\E[2]}}
      {\TPairOrd{\T[1]}{\T[2]}}
      {\Eff[1]}
  }
}

\newcommand*\RuleTLetUnit{
  \inferrule[T-LetUnit]{
    \HasType[{\Q[1]}]
      {\Ctx}
      {\E[1]}
      {\BUnit}
      {\Eff[1]} \\
    \HasType[{\Q[2]}]
      {\CtxPattern{\CtxEmpty}}
      {\E[2]}
      {\T[2]}
      {\Eff[2]} \\
      \Eff[1] = \EffImpure \implies \IsLeftPat~\CtxPatternRaw
  }{
    \HasType[{\Q[1] \plub \Q[2]}]
      {\CtxPattern{\Ctx}}
      {\ESeq{\E[1]}{\E[2]}}
      {\T[2]}
      {(\Eff[1] \EffJoin \Eff[2])}
  }
}

\newcommand*\RuleTLet{
  \inferrule[T-Let]{
    \HasType[{\Q[1]}]
      {\Ctx}
      {\E[1]}
      {\T[1]}
      {\Eff[1]} \\
    \HasType[{\Q[2]}]
      {\CtxPattern{\CtxVar x{{\T[1]}}}}
      {\E[2]}
      {\T[2]}
      {\Eff[2]} \\
      \Eff[1] = \EffImpure \implies \IsLeftPat~\CtxPatternRaw
  }{
    \HasType[{\Q[1] \plub \Q[2]}]
      {\CtxPattern{\Ctx}}
      {\ELet x{\E[1]}{\E[2]}}
      {\T[2]}
      {(\Eff[1] \EffJoin \Eff[2])}
  }
}

\newcommand*\RuleTLetUnord{
  \inferrule[T-LetUnord]{
    \HasType[{\Q[1]}]
      {\Ctx}
      {\E[1]}
      {\TPairUnord {\T[1]}{\T[2]}}
      {\Eff[1]} \\
    \HasType[{\Q[2]}]
      {\CtxPattern{\CtxExtUnord{\CtxVar x{\T[1]}}{\CtxVar y{\T[2]}}}}
      {\E[2]}
      {\T[3]}
      {\Eff[2]} \\
      \Eff[1] = \EffImpure \implies \IsLeftPat~\CtxPatternRaw
  }{
    \HasType[{\Q[1] \plub \Q[2]}]
      {\CtxPattern{\Ctx}}
      {\ELetPair xy{\E[1]}{\E[2]}}
      {\T[3]}
      {(\Eff[1] \EffJoin \Eff[2])}
  }
}

\newcommand*\RuleTLetOrd{
  \inferrule[T-LetOrd]{
    \HasType[{\Q[1]}]
      {\Ctx}
      {\E[1]}
      {\TPairOrd {\T[1]}{\T[2]}}
      {\Eff[1]} \\
    \HasType[{\Q[2]}]
      {\CtxPattern{\CtxExt{\CtxVar x{\T[1]}}{\CtxVar y{\T[2]}}}}
      {\E[2]}
      {\T[3]}
      {\Eff[2]} \\
      \Eff[1] = \EffImpure \implies \IsLeftPat~\CtxPatternRaw
  }{
    \HasType[{\Q[1]\plub\Q[2]}]
      {\CtxPattern{\Ctx}}
      {\ELetPair xy{\E[1]}{\E[2]}}
      {\T[3]}
      {(\Eff[1] \EffJoin \Eff[2])}
  }
}

\newcommand*\RuleTInLeft{
  \inferrule[T-Inj]{
    \HasType[\Q]
    {\Ctx}
    {\E}
    {\T[i]}
    {\Eff}
  }{
    \HasType[\Q]
    {\Ctx}
    {\EInj{i} \E}
    {\TSum{\T[1]}{\T[2]}}
    {\Eff}
  }
}

\newcommand*\RuleTCaseSum{
  \inferrule[T-CaseSum]{
    \HasType[\Q]
    {\Ctx}
    {\E}
    {\TSum{\T[1]}{\T[2]}}
    {\Eff}
    \\
    \HasType[{\Q[1]}]
    {\CtxPattern{\CtxVar{\EVar_1}{\T[1]}}}
    {\E[1]}
    {\T}
    {\Eff[1]}
    \\
    \HasType[{\Q[2]}]
    {\CtxPattern{\CtxVar{\EVar_2}{\T[2]}}}
    {\E[2]}
    {\T}
    {\Eff[2]} \\
    \Eff = \EffImpure \implies \IsLeftPat~\CtxPatternRaw
  }{
    \HasType[{\Q \plub \Q[1]}]
    {\CtxPattern{\Ctx}}
    {\ECase\E{x_1}{\E[1]}{x_2}{\E[2]}}
    {\T}
    {(\Eff \EffJoin \Eff[1] \EffJoin \Eff[2])}
  }
}

\newcommand*\RuleTWeakC{
  \inferrule[T-WeakC]{
    \Unr\Ctx_2 \\
    \HasType
      {\CtxExt{\Ctx_1}{\Ctx_2}}
      {\E}
      {\T}
      {\Eff} 
  }{
    \HasType
      {\Ctx_1}
      {\E}
      {\T}
      {\Eff}
  }
}

\newcommand*\RuleTWeakE{
  \inferrule[T-WeakE]{
    \HasType
      {\Ctx}
      {\E}
      {\T}
      {\EffPure} 
  }{
    \HasType
      {\Ctx}
      {\E}
      {\T}
      {\EffImpure}
  }
}

\newcommand*\RuleTWeakCE{
  \inferrule[T-WeakCE]{
    \Ctx[2] \preccurlyeq \Ctx[1]
    \\
    \Eff[1] \le \Eff[2]
    \\
    \HasType[\Q]
      {\Ctx_1}
      {\E}
      {\T}
      {\Eff[1]}
  }{
    \HasType[\Q]
      {\Ctx_2}
      {\E}
      {\T}
      {\Eff[2]}
  }
}

\newcommand*\RuleTMatch{
  \inferrule[T-MatchLeft]{
    \HasType
      {\Ctx_1}
      {\E}
      {\SBranch{\ell:\S_\ell}{\ell\in L}}
      {\Eff}
    \\
    \HasType
      {\Ctx_2}
      {\E_\ell}
      {\TArr{\Mob}{\Eff[\ell]}{\S_\ell}{\T}}
      {\Eff[\ell]}
  }{
    \HasType
      {\CtxSeq{\Ctx_1}{\Ctx_2}}
      {\EMatch{\E}{\ell:\E_\ell}{\ell\in L}}
      {\T}
      {(\Eff \EffJoin \EffJoin\{ \Eff[\ell]\}_{\ell\in L})}
  }
}

\newcommand*\RuleTMatchLin{
  \inferrule[T-MatchLin]{
    \HasType
      {\Ctx_1}
      {\E}
      {\SBranch{\ell:\S_\ell}{\ell\in L}}
      {\Eff}
    \\
    \HasType
      {\Ctx_2}
      {\E_\ell}
      {\TArr{\Mob}{\Eff[\ell]}{\S_\ell}{\T}}
      {\Eff[\ell]}
  }{
    \HasType
      {\CtxPar{\Ctx_1}{\Ctx_2}}
      {\EMatch{\E}{\ell:\E_\ell}{\ell\in L}}
      {\T}
      {(\Eff \EffJoin \EffJoin\{ \Eff[\ell]\}_{\ell\in L})}
  }
}

%% EXISTENTIAL TARGET CALCULUS

\newcommand*\RuleTPack{
  \inferrule[T-Pack]{
    \HasTargetType[\Q]
    \Ctx
    {\E}
    {\T{[{\S}/\alpha]}}
  }{
    \HasTargetType[\Q]
    \Ctx
    {\EPack{\S}{\E}{\TExists[\pr{S}]\alpha\T}}
    {\TExists[\pr{S}]\alpha\T}
  }
}

\newcommand*\RuleTUnpack{
  \inferrule[T-Unpack]{
    \HasTargetType[{\Q[1]}]
    {\Ctx}
    {\E[1]}
    {\TExists[\O]\alpha{\T[1]}}
    \\
    \HasTargetType[{\Q[2]}]
    {\CtxExtTarget{\CtxExtTarget\Ctx\alpha}{\CtxVar x {\T[1]}}}
    {\E[2]}
    {\T[2]}
    \\
    \alpha \notin \T[2]
  }{
    \HasTargetType[{\Q[1] \plub \Q[2]}]
    \Ctx
    {\EOpen\alpha x {\E[1]} {\E[2]}}
    {\T[2]}
  }
}

%%% Local

%%% Local Variables:
%%% mode: latex
%%% TeX-master: "../main"
%%% End:

% Process typing

\newcommand*\RuleTExpr{
  \inferrule[TP-Expr]{
    \HasType[\pbot]
      \Ctx
      \E
      \BUnit
      \EffImpure
  }{
    \ProcHasType
    \Ctx
    {\PExp{\E}}
  }
}

\newcommand*\RuleTPar{
  \inferrule[TP-Par]{
    \ProcHasType{\Ctx[1]}{\Proc[1]} \\
    \ProcHasType{\Ctx[2]}{\Proc[2]}
  }{
    \ProcHasType
    {\CtxExtUnord{\Ctx[1]}{\Ctx[2]}}
    {\PPar{\Proc[1]}{\Proc[2]}}
  }
}

\newcommand*\RuleTRes{
  \inferrule[TP-Res]{
    \ProcHasType
    { \CtxExtUnord\Ctx{\CtxExtUnord{\Ctx[1]}{\Ctx[2]}}
    }
    \Proc  \\
    \S \vdash \Bind[1] \leadsto \Ctx[1] \\
    \SDual\S \vdash \Bind[2] \leadsto \Ctx[2]
  }{
    \ProcHasType
    {\Ctx}
    {\PNew{\Bind[1]}{\Bind[2]}\Proc}
  }
}

\newcommand*\RuleBindEmp{
  \inferrule[B-Emp]{}{
    \SSkip \vdash \BEmp \leadsto \CtxEmpty
  }
}

\newcommand*\RuleBindSeq{
  \inferrule[B-Seq]{
    \S[1] \vdash \Bind[1] \leadsto \Ctx[1] \\
    \S[2] \vdash \Bind[2] \leadsto \Ctx[2]
  }{
    \SSeq{\S[1]}{\S[2]} \vdash \BSeq{\Bind[1]}{\Bind[2]} \leadsto \CtxExt{\Ctx[1]}{\Ctx[2]}
  }
}

\newcommand*\RuleBindPar{
  \inferrule[B-Par]{
    \SSeq{\S[1]}{\SRet} \vdash \Bind[1] \leadsto \Ctx[1] \\
    \SSeq{\SAcq}{\S[2]} \vdash \Bind[2] \leadsto \Ctx[2]
  }{
    \SSeq{\S[1]}{\S[2]} \vdash \BPar{\Bind[1]}{\Bind[2]} \leadsto \CtxExtUnord{\Ctx[1]}{\Ctx[2]}
  }
}

\newcommand*\RuleBindVar{
  \inferrule[B-Var]{}{
    \S \vdash \EVar \leadsto \CtxVar\EVar\S
  }
}

%%% Local Variables:
%%% mode: latex
%%% TeX-master: "../main"
%%% End:

% Reduction

\newcommand*\RuleExpRedApp{
  \inferrule[E-App]{}{
    \EApp{(\EAbs \EVar\E)}\V \Reduces \Subs \E\V\EVar
  }
}

\newcommand*\RuleExpRedSeq{
  \inferrule[E-Seq]{}{
    \ESeq \CUnit \E \Reduces  \E
  }
}

\newcommand*\RuleExpRedLet{
  \inferrule[E-PairElim]{}{
    \ELet {\EPair xy} {\EPair uv} \E \Reduces \Subs {\Subs \E vx} uy
  }
}

\newcommand*\RuleExpRedCaseLeft{
  \inferrule[E-SumElim]{}{
    \ECase{(\EInj i\V)}{x_1}{\E[1]}{x_2}{\E[2]} \Reduces \Subs {\E[i]} \V {x_i}
  }
}

% \newcommand*\RuleExpRedCaseRight{
%   \inferrule[E-SumElim]{}{
%     \ECase{(\EInj2\U)}{x}{\E[1]}{y}{\E[2]} \Reduces \Subs {\E[2]} \U y
%   }
% }

\newcommand*\RuleExpRedCtx{
  \inferrule[E-Ctx]{
    \E[1] \Reduces \E[2]
  }{
    \EC[\E[1]] \Reduces \EC[\E[2]]
  }
}

\newcommand*\RuleProcRedExp{
  \inferrule[R-Exp]{
    \E[1] \Reduces \E[2]
  }{
    \FCapp {\E[1]} \Reduces \FCapp{\E[2]}
  }
}

\newcommand*\RuleProcRedClose{
  \inferrule[R-Close]{}{
    \PNew xy {(\PPar { {\FCapp[1]{\EApp\CTerm x}}} { {\FCapp[2]{\EApp\CWait y}}})} 
    \Reduces 
    \PPar { {\FCapp[1]{\CUnit}}} { {\FCapp[2]{\CUnit}}}
  }
}

\newcommand*\RuleProcRedSplit{
  \inferrule[R-RSplit]{}{
    \PNew {\BSeq{\BindGroup[1]}{\BSeq{\EVar} {\BindGroup[2]}}}{\BindGroup[3]}{ (\PPar{\FCapp{\EApp \CSSplit {x}}}\Proc)}
    \Reduces 
    \PNew {\BPar{\BSeq{\BindGroup[1]}{\EVar[1]}}{\BSeq{\EVar[2]}{\BindGroup[2]}}}{\BindGroup[3]} { ({ \PPar{\FCapp{\EPair{\EVar[1]}{\EVar[2]}}}\Proc})}
  }
}

\newcommand*\RuleProcRedSplitLocal{
  \inferrule[R-LSplit]{}{
    \PNew {\BSeq{\BindGroup[1]}{\BSeq{\EVar} {\BindGroup[2]}}}{\BindGroup[3]}{ (\PPar{\FCapp{\EApp \CSplit {x}}}\Proc)}
    \Reduces 
    \PNew {\BSeq{\BindGroup[1]}{\BSeq{\BSeq{\EVar[1]}{\EVar[2]}} {\BindGroup[2]}}}{\BindGroup[3]} { ({ \PPar{\FCapp{\EPairOrd{\EVar[1]}{\EVar[2]}}}\Proc})}
  }
}

\newcommand*\RuleProcRedDrop{
  \inferrule[R-Drop]{}{
    \PNew {\BSeq{\EVar}{\BindGroup[1]}}{\BindGroup[2]}{ (\PPar{\FCapp{\EApp \CDrop {\EVar}}}\Proc)}
    \Reduces 
    \PNew {\BindGroup[1]}{\BindGroup[2]} { ({ \PPar{\FCapp{\CUnit}}\Proc})}
  }
}

\newcommand*\RuleProcRedAcquire{
  \inferrule[R-Acquire]{}{
    \PNew {\BSeq{\EVar}{\BindGroup[1]}}{\BindGroup[2]}{ (\PPar{\FCapp{\EApp \CAcq{\EVar}}}\Proc)}
    \Reduces 
    \PNew {\BSeq{\EVar}{\BindGroup[1]}}{\BindGroup[2]} { ({ \PPar{\FCapp{\EVar}}\Proc})}
  }
}


\newcommand*\RuleProcRedCom{
  % \inferrule[R-Com]{}{
  %   \PNew xy (\PPar { {\FC_1[\EApp \CSend {\EPair\V x}]}} { {\FC_2[\EApp\CRecv y]}}) 
  %   \Reduces 
  %   \PPar { {\FC_1[\CUnit]}} { {\FC_2[\V]}}
  % }
  % \\
  \inferrule[R-Com]{}{
    \PNew {\BSeq x{\BindGroup[1]}}{\BSeq y {\BindGroup[2]}}{
      (\PPar{\PPar { {\FCapp[1]{\EApp \CSend {(\EPair\V x)}}}} {
          {\FCapp[2]{\EApp\CRecv y}}}} \Proc)} 
    \Reduces
    \PNew {\BindGroup[1]}{\BindGroup[2]}
    {(\PPar{\PPar{ {\FCapp[1]{\CUnit}}} { {\FCapp[2]{\V}}}} \Proc)}
  }
}

\newcommand*\RuleProcRedChoice{
  \inferrule[R-Choice]{
    k \in L
  }{
    \PNew xy ({\PPar {\PExp {E_1[\CSelect{k} x]}}{\PExp{\EMatch{\EVar}{\ell:\E[\ell]}{\ell\in L}}}})
    \Reduces
    \PNew xy (\PPar {\PExp {E_1[x]}} {\PExp {E_2[\EApp {\E[k]} y]}})
  }
}

\newcommand*\RuleProcRedNew{
  \inferrule[R-New]{}{
    { {\FCapp{\EApp \CNew \CUnit}}}
    \Reduces 
    \PNew xy { {\FCapp{\EPair xy}}}
  }
}

\newcommand*\RuleProcRedLink{
  \inferrule[R-Link]{}{
    \PNew xy {(\PPar{{\FCapp{\EApp\CLink{\EPair w x}}}}{\Proc})}
    \Reduces 
    \PPar{ {\FCapp{\CUnit}}}{\Subs \Proc wy}
  }
}

\newcommand*\RuleProcRedFork{
  \inferrule[R-Fork]{}{
    { {\FCapp{\EApp \CFork \V}}}
    \Reduces 
    \PPar { {\FCapp{\CUnit}}} {\PExp {\EApp\V\CUnit}}
  }
}

\newcommand*\RuleProcRedStruct{
  \inferrule[R-Struct]{
    \Proc[1] \equiv \Proc[1]' \\
    \Proc[1]' \Reduces \Proc[2]' \\
    \Proc[2]' \equiv \Proc[2]
  }{
    \Proc[1] \Reduces \Proc[2]
  }
}

\newcommand*\RuleProcRedPar{
  \inferrule[R-Par]{
    \Proc[1] \Reduces \Proc[2]
  }{
    \PPar {\Proc[1]}{ \Proc[3]} \Reduces \PPar {\Proc[2]}{ \Proc[3]}
  }
}

\newcommand*\RuleProcRedBind{
  \inferrule[R-Bind]{
    \Proc[1] \Reduces \Proc[2]
  }{
    \PNew{\BindGroup[1]}{\BindGroup[2]} {\Proc[1]} \Reduces \PNew {\BindGroup[1]}{\BindGroup[2]}{\Proc[2]}
  }
}

%     \qquad
%     \axiom{\rulenameexpredLet}{
%       \lete \ell xL{\recorde \ell vL}{e} \reduces \subs
%       {v_\ell}{x_\ell}e_{\ell\in L}}
%     \\
%     \axiom{\rulenameexpredUnitElim}{\unlete {\unite}{\unite}e \reduces e}
%     \qquad
%     \axiom{\rulenameexpredTApp}{(\tappe{\tabse a \_ v)}{T} \reduces \subs Tav}
%     \\
%     \infrule{\rulenameexpredCase}
%     {k\in L}
%     {\casee{\injecte k v}{\recordp \ell e L}  \reduces e_kv}
%     \qquad
%     \infrule{\rulenameexpredCtx}
%     {e \reduces e'}
%     {E[e] \reduces E[e']}
%   \end{gather*}
%   % 
%   \declrel{Process reduction}{$\isRed pp$}
%   \begin{gather*}
%     \infrule{\rulenameprocredExp}
%     {e \reduces e'} 
%     {\PROC{e} \reduces \PROC{e'}}
%     \quad
%     \axiom{\rulenameprocredFork}{\PROC{E[\appe{\tappe\forkk\_} e]} \reduces
%       \PROC{E[\unite[\un]]} \PAR \PROC e}
%     \quad
%     \axiom{\rulenameprocredNew}{\PROC{E[\newe \_]} \reduces\NU xy\PROC{E[(x,y)]}}
%     \\
%     \axiom{\rulenameprocredMsg}{
%       \NU xy(\PROC{E_1[\sendk[\_]v[\_]x]}
%       \PAR \PROC{E_2[\receivek[\_][\_]y})
%       % \NU xy(\PROC{E_1[\appe{\tappe{\appe{\tappe\sendk\_}v}\_}x]}
%       % \PAR  \PROC{E_2[\appe{\tappe{\tappe\receivek\_}\_} y]})
%       \reduces\NU xy(\PROC{E_1[x]} \PAR \PROC{E_2[(v,y)]})}
%     \\
%     \infrule{\rulenameprocredCh}
%     {k\in L}
%     {\NU xy(\PROC{E_1[\selecte k x]} \PAR \PROC{E_2[\matche
%         y{\recordp \ell e L}]}) \reduces
%       \hspace{10em}\\\hspace{23em}
%       \NU xy(\PROC{E_1[x]} \PAR \PROC{E_2[e_ky]})
%     }
%     \\
%     \infrule{\rulenameprocredPar}{p \reduces p'}{p\PAR q \reduces p'\PAR q}
%     \qquad
%     \infrule{\rulenameprocredBind}{p \reduces p'}{\NU xyp \reduces \NU xyp'}
%     \qquad
%     \infrule{\rulenameprocredCong}{p \equiv q\\ q \reduces q'}{p \reduces q'}
%   \end{gather*}
%   % not needed because the evaluation contexts do not bind any variables
%   % Context $E_1$ (resp.~$E_2$, resp.~$E$) does not bind~$x$ (resp.~$y$,
%   % resp.~$x$ and~$y$). 
%   % \\
%   % Dual $\NU xy$-rules for $\sendk/\receivek$ and $\selectk/\casek$
%   % omitted. % Use structural congruence
%   \caption{Reduction}
%   \label{fig:reduction}
% \end{figure}

%%% Local Variables:
%%% mode: latex
%%% TeX-master: "../main"
%%% End:

% Congruence

\newcommand*\RuleProcCongRefl{
  \inferrule[C-Refl]{}{
    \Proc \Congruent \Proc
  }
}

\newcommand*\RuleProcCongParComm{
  \inferrule[C-ParComm]{}{
    \PPar{\Proc[1]}{\Proc[2]} \Congruent \PPar{\Proc[2]}{\Proc[1]}
  }
}

\newcommand*\RuleProcCongParAssoc{
  \inferrule[C-ParAssoc]{}{
    \PPar{\Proc[1]}{(\PPar{\Proc[2]}{\Proc[3]})} \Congruent \PPar{(\PPar {\Proc[1]}{\Proc[2]})}{\Proc[3]}
  }
}

\newcommand*\RuleProcCongTrans{
  \inferrule[C-Trans]{
    \Proc[1] \Congruent \Proc[2] \\
    \Proc[2] \Congruent \Proc[3]
  }{
    \Proc[1] \Congruent \Proc[3]
  }
}

\newcommand*\RuleProcCongParUnit{
  \inferrule[C-ParUnit]{}{
    \PPar{\PExp{\CUnit}}{\Proc} \Congruent \Proc
  }
}

\newcommand*\RuleProcCongResSwap{
  \inferrule[C-ResSwap]{}{
    \PNew{\BindGroup[1]}{\BindGroup[2]}\Proc \Congruent \PNew{\BindGroup[2]}{\BindGroup[1]}\Proc
  }
}

\newcommand*\RuleProcCongResComm{
  \inferrule[C-ResComm]{
    (\Fv(\BindGroup[1]) \cup \Fv(\BindGroup[2])) \cap (\Fv(\BindGroup[3]) \cup \Fv(\BindGroup[4])) = \emptyset
  }{
    \PNew{\BindGroup[1]}{\BindGroup[2]}{\PNew{\BindGroup[3]}{\BindGroup[4]}\Proc}
    \Congruent
    \PNew{\BindGroup[3]}{\BindGroup[4]}{\PNew{\BindGroup[1]}{\BindGroup[2]}\Proc} 
  }
}

\newcommand*\RuleProcCongExtend{
  \inferrule[C-Extend]{
    \Fv(\Proc[1]) \cap (\Fv(\BindGroup[1]) \cup \Fv(\BindGroup[2])) = \emptyset
  }{
    \PPar{\Proc[1]}{\PNew{\BindGroup[1]}{\BindGroup[2]}{\Proc[2]}}
    \Congruent
    {\PNew{\BindGroup[1]}{\BindGroup[2]}{(\PPar{\Proc[1]}{\Proc[2]})}}
  }
}




%%% Local Variables:
%%% mode: latex
%%% TeX-master: "../main"
%%% End:


\newtheorem{lemma}{Lemma}

\title{Context-Free Session Types with Borrowing}
\date{August 28, 2025}

\begin{document}
\maketitle

\begin{figure}[tp]
  \begin{itemize}
  \item prove subject reduction (start: April 24)
  \item sketch translation, prove simulation (start: April 24)
  \item target FreeST according to I\&C paper; encode existentials by
    universals
  \item maybe: bisimulation
  \item later: add recursive types and recursion to source language
  \item later: add explicit polymorphism as in FreeST
  \end{itemize}
  \caption{The Plan}
  \label{fig:the-plan}
\end{figure}


The next couple of lemmas are needed to prove that
expression reduction preserves typing.

\begin{lemma}
  If $\HasType {\Ctx} {\V} {\T} {\Eff}$
  then $\Eff = \EffPure$.
\end{lemma}
\begin{proof}
  By induction on the derivation of $\HasType {\Ctx} {\V} {\T} {\Eff}$.
\end{proof}

\begin{lemma}[Substitution]\
  \begin{itemize}
    \item If $\HasType {\Ctx[1]} {\V} {\T[1]} {\Eff[1]}$ and
    $\HasType {\CtxExtUnord{\CtxVar x{\T[1]}}{\Ctx[2]}} {\E} {\T[2]} {\Eff[2]}$
    then 
    $\HasType {\CtxExtUnord{\Ctx[1]}{\Ctx[2]}} {\Subs\E\V x} {\T[2]} {\Eff[2]}$
    \item If $\HasType {\Ctx[1]} {\V} {\T[1]} {\Eff[1]}$ and
    $\HasType {\CtxExt{\CtxVar x{\T[1]}}{\Ctx[2]}} {\E} {\T[2]} {\Eff[2]}$
    then 
    $\HasType {\CtxExt{\Ctx[1]}{\Ctx[2]}} {\Subs\E\V x} {\T[2]} {\Eff[2]}$
    \item If $\HasType {\Ctx[1]} {\V} {\T[1]} {\Eff[1]}$
    and
    $\HasType
    {\CtxPattern{\CtxExtUnord{\CtxVar x{\T[1]}}{\Ctx[2]}}}
    {\E}
    {\T[2]}
    {\Eff[2]}$ then 
    $\HasType {\CtxPattern{\CtxExtUnord{\Ctx[1]}{\Ctx[2]}}} {\Subs\E\V x} {\T[2]} {\Eff[2]}$
  \item If $\HasType {\Ctx[1]} {\V} {\T[1]} {\Eff[1]}$
    and
    $\HasType
    {\CtxPattern{\CtxExt{\CtxVar x{\T[1]}}{\Ctx[2]}}}
    {\E}
    {\T[2]}
    {\Eff[2]}$ then 
    $\HasType {\CtxPattern{\CtxExt{\Ctx[1]}{\Ctx[2]}}} {\Subs\E\V x} {\T[2]} {\Eff[2]}$
  \end{itemize}
\end{lemma}
\begin{lemma}[Substitution]
  \label{lem:substitution}
  If 
  $\HasType {\Ctx[0]} {\E[0]} {\T[0]} {\EffPure}$
  and
  $\MobMobile (\Ctx[0])$
  and
  $\HasType
    {\CtxPattern[0]{\CtxVar y{\T[0]}}}
    {\E}
    {\T}
    {\Eff}$
    implies
    $\HasType {\CtxPattern[0]{{\Ctx[0]}}} {\Subs\E{\E[0]} y} {\T}
    {\Eff}$, for disjoint $\Ctx[0]$ and $\CtxPatternRaw_0$.
  \end{lemma}
  PT: somehow we need to make sure that $\Ctx[0]$ is independent so
  that it can safely be moved around in the context.
\begin{proof}
  By rule induction on the derivation of  $\HasType
  {\CtxPattern[0]{\CtxVar x{\T[0]}}} {\E} {\T} {\Eff}$.

  \textbf{Case} $\RuleTConst$. n.a.

  \textbf{Case} $\RuleTVar$. Immediate for $x=y$ and empty $\CtxPattern[0]{}$.

  \textbf{Case} $\RuleTAbsLin$. Here $x\ne y$; by variable convention, assume
  $x \notin dom (\Ctx[0])$; by induction,
  $\HasType
  {\CtxExtUnord{\CtxPattern[0]{{\Ctx[0]}}}{\CtxVar\EVar{\T[1]}}}
  {\Subs\E{\E[0]} y} {\T[2]} {\Eff}$; by assumption, $\MobMobile
  (\Ctx[0])$ which implies $\Mob (\CtxPattern[0]{{\Ctx[0]}})$; conclude
  by \TirName{T-AbsLin}.

  \textbf{Case} \TirName{T-AbsLeft} and \TirName{T-AbsRight} are
  analogous.

  \textbf{Case} $\RuleTAppLin$. \\
  Either $y\in \E[1]$ or $y \in \E[2]$.
  If $y\in \E[1]$, then $\Ctx[1]$ has the form $\CtxPattern[0]{\CtxVar y
    {\T[0]}}$ and induction yields
  $\HasType {\CtxPattern[0]{\Ctx[0]}} {\Subs{\E[1]}{\E[0]}y}
  {(\TArr{\Mob}{\DirUnord,{\Eff[3]}}{\T[1]}{\T[2]})}
  {\Eff[1]}$. Conclude by applying \TirName{TApp-Lin}. 

  If $y\in \E[2]$, argue analogously.

  \textbf{Case} \TirName{T-AppLeft} and \TirName{T-AppRight} are
  analogous.

  \textbf{Case} \TirName{T-PairUnord} and \TirName{T-PairOrd} are analogous.

  \textbf{Case} $\RuleTLet$.

  Either $y\in\E[1]$ or $x\ne y$ and $y\in\E[2]$.

  If $y\in\E[1]$, then $\Ctx$ has the form $\CtxPattern[1]{\CtxVar y
    {\T[0]}}$ and $\CtxPattern[0]{} = \CtxPattern{\CtxPattern[1]{}}$.
  By induction, $\HasType {\CtxPattern[1]{\Ctx[0]}} {\E[1]} {\T[1]}
  {\Eff[1]}$ and applying \TirName{T-Let} yields a typing in context
  $\CtxPattern[0]{\Ctx[0]}$.

  If $x \ne y \in \E[2]$, then $y$ must appear in $\CtxPatternRaw$, so
  that $\CtxPattern{\Ctx} = \CtxPattern[0]{\CtxVar y {\T[0]}}[{\Ctx}]$, for
  some 2-hole pattern $\CtxPatternRaw_0$. By induction using the
  pattern $\CtxPattern[0]{\CtxVar y {\T[0]}}[{\CtxVar x {\T[1]}}]$, we
  obtain a judgment in context $\CtxPattern[0]{\Ctx[0]}[{\CtxVar x {\T[1]}}]$
 and applying \TirName{T-Let}
  we obtain a typing in context $\CtxPattern[0]{\Ctx[0]}[{\Ctx}]$. The
  $\IsLeftPat$ property is not affected as it refers to the position
  of $x$ in the context.

  \textbf{Case} $\RuleTLetUnit$.

  Either $y \in \E[1]$ or $y \in \E[2]$.

  If $y \in \E[1]$, then the argument is analogous to \TirName{T-Let}.

  If $y \in \E[2]$, then $y$ must appear in $\CtxPatternRaw$, so
  that $\CtxPattern{\Ctx} = \CtxPattern[0]{\CtxVar y {\T[0]}}[{\Ctx}]$, for
  some 2-hole pattern $\CtxPatternRaw_0$. By induction using the
  pattern $\CtxPattern[0]{\CtxVar y {\T[0]}}[\CtxEmpty]$, we
  obtain a judgment in context $\CtxPattern[0]{\Ctx[0]}[\CtxEmpty]$
 and applying \TirName{T-LetUnit}
 we obtain a typing in context $\CtxPattern[0]{\Ctx[0]}[{\Ctx}]$.

 \textbf{Case} \TirName{T-LetUnit} and \TirName{T-LetOrd} are analogous
 to \TirName{T-Let} and \TirName{T-LetUnit}.

 \textbf{Case} \TirName{T-Inj}. Immediate by induction.

 \textbf{Case} \TirName{T-CaseSum}. Analogous to \TirName{T-Let}.

 \textbf{Case} $\RuleTWeakCE$.

 TODO!
\end{proof}

\begin{lemma}[Expression reduction preserves typing]
  If $\E[1] \Reduces \E[2]$ and 
  $\HasType
      {\Ctx}
      {\E[1]}
      {\T}
      {\Eff[1]}$
  then 
  $\HasType
      {\Ctx}
      {\E[2]}
      {\T}
      {\Eff[2]}$
  for some $\Eff[2] \leq \Eff[1]$.
\end{lemma}

The following is an auxiliary lemma to prove that 
process reduction preserves typing.

\begin{lemma}
    $\HasType
      {\CtxExtUnord {\Ctx[1]}{\Ctx[2]}}
      {\EC[\E]}
      {\T[1]}
      {\Eff[1]}$ 
    if and only if
    $\HasType
      {\Ctx[1]}
      {\E}
      {\T[2]}
      {\Eff[2]}$
    and 
    $\HasType
      {\CtxExtUnord {\CtxVar x{\T[2]}}{\Ctx[2]}}
      {\EC[x]}
      {\T[1]}
      {\Eff[1]}$ 
\end{lemma}

\begin{lemma}
  \label{lemma:inversion-evaluation-context}
  If $\HasType{\Ctx}{\EC[\E]}{\T}{\Eff}$, then there exists $\Ctx'$,
  $\T'$, and $\Eff'$ such that $\HasType{\Ctx'}{\E}{\T'}{\Eff'}$.
\end{lemma}
\begin{proof}
  By induction on $\EC$.
\end{proof}

\begin{lemma}
  \label{lemma:inversion-process-evaluation-context}
  If $\ProcHasType{\Ctx_0} {\FCapp{\E}}$, then there exists $\Ctx$,
  $\T$, and $\Eff$ such that $\HasType{\Ctx}{\E}{\T}{\Eff}$.
\end{lemma}
\begin{proof}
  It must be that $\FC = \PExp \EC$, for some evaluation context $\EC$.
  By inversion of the process typing,
  $\HasType{\Ctx_0}{\EC[\E]}{\TBUnit}{\EffImpure}$.
  By Lemma~\ref{lemma:inversion-evaluation-context}, there is  $\Ctx$,
  $\T$, and $\Eff$ such that $\HasType{\Ctx}{\E}{\T}{\Eff}$.
\end{proof}

\clearpage
\begin{lemma}[Process reduction preserves typing]
  If $\Proc[1] \Reduces \Proc[2]$ and 
  $\ProcHasType {\Ctx} {\Proc[1]}$
  then 
  $\ProcHasType {\Ctx} {\Proc[2]}$.
\end{lemma}
\begin{proof}
  \textbf{Case} \TirName{R-Exp}.

  \textbf{Case} $\left\{\RuleProcRedNew\right.$.\\
  As $\ProcHasType\Ctx {\FCapp{\EApp \CNew \CUnit}}$, there must be
  some $\Ctx'$ such that
  $\HasType{\Ctx'}{\EApp \CNew \CUnit}{\T}{\Eff}$ (by Lemma~\ref{lemma:inversion-process-evaluation-context}). \\
  By inversion, we obtain $\Ctx' = \CtxEmpty$ and $\T =
  \TPairUnord{(\SSeq{\SAcq}{\S})}{(\SSeq{\SAcq}{\Dual~\S})}$. \\
  Let $\Ctx'' = \CtxPar{\CtxVar x{(\SSeq{\SAcq}{\S})} }{\CtxVar y
    {\SSeq{\SAcq}{\Dual~\S}}}$. \\
  We have
  $\HasType{\Ctx''}{\EPair
    xy}{\TPairUnord{(\SSeq{\SAcq}{\S})}{(\SSeq{\SAcq}{\Dual~\S})}}{\EffPure}$.
  \\
  By lemma XXX, we have
  $\ProcHasType{\CtxPar{\Ctx}{\Ctx''}}{\FCapp{\EPair xy}}$.
  \\
  Let $\Ctx[1] =\CtxVar x{(\SSeq{\SAcq}{\S})}$ and $\Ctx[2] = \CtxVar
  y {\SSeq{\SAcq}{\Dual~\S}}$ so that $\Ctx'' =
  \CtxPar{\Ctx[1]}{\Ctx[2]}$.
  \\
  Clearly, $\S \vdash {\BPar{\BEmp}{x}} \leadsto \Ctx[1]$ and
  $\Dual~\S \vdash {\BPar{\BEmp}{y}} \leadsto \Ctx[2]$ hold.
  \\
  Conclude with rule \TirName{TP-Res} and obtain
  $\ProcHasType {\Ctx}
  {\PNew {\BPar{\BEmp}{x}}{\BPar{\BEmp}{y}}{\FCapp{\EPair xy}}}$.

  

  \textbf{Case} $\left\{ \RuleProcRedFork \right.$.\\
  As $\ProcHasType\Ctx {\FCapp{\EApp \CFork \V}}$, there must be some
  $\Ctx'$ such that
  $\HasType{\Ctx'}{\EApp \CFork \V}{\T}{\Eff}$ (by
  Lemma~\ref{lemma:inversion-process-evaluation-context}). \\
  By inversion, we obtain
  $\HasType{\Ctx'}\V{\TArr{\MobMobile}{\DirUnord,\EffImpure}\BUnit\BUnit}\EffPure$,
  so that $\T = \BUnit$ and $\Eff = \EffImpure$ and $\Ctx =
  \CtxPar{\Ctx'}{\Ctx''}$ (because the function is $\DirUnord$ its
  application must do a parallel split), for some $\Ctx''$.
  \\
  NEED A LEMMA to conclude $\ProcHasType{\Ctx''}{\FCapp{\CUnit}}$.
  \\
  Clearly, $\ProcHasType{\Ctx'}{\PExp {\EApp\V\CUnit}}$, so that
  $\ProcHasType{\CtxPar{\Ctx'}{\Ctx''}}{\PPar{\FCapp{\CUnit}}{\PExp {\EApp\V\CUnit}}}$.

  Case \TirName{R-Com}.

  Case \TirName{R-RSplit}.

  Case \TirName{R-LSplit}.

  Case \TirName{R-Drop}.

  Case \TirName{R-Acquire}.

  Case \TirName{R-Close}.

  Case \TirName{R-Par}.

  Case \TirName{R-Bind}.

  Case \TirName{R-Struct}.
\end{proof}

% SYNTAX OF EXPS

\begin{figure}
  \begin{align*}
    \C \grmeq
      & \CUnit
        % \grmor \CLink
        \grmor \CNew
        \grmor \CFork
        \grmor \CTerm 
        \grmor \CWait
        \grmor \CSend
        \grmor \CRecv
    \\&
    \grmor \CSplit \grmor \CSSplit
    \grmor \CDrop \grmor \CAcq
        \tag{Constants} \\
    \E \grmeq
      & \C \grmor \EVar \grmor \EAbs\EVar\E \grmor \EApp\E\E 
        \grmor \ESeq\E\E
        \grmor \EPair\E\E
        \grmor \ELetPair{\EVar}{\EVar}{\E}{\E} \\
      &  \grmor \EInj i \E 
        \grmor\ECase\E{\EVar}{\E}{\EVar}{\E} \\
      % & \grmor \EMatch{\E}{\ell:\E_\ell}{\ell\in L}
      % \grmor \EFork\E \grmor \ELet\EVar\E\E
        \tag{Expressions}
  \end{align*}
  \caption{Syntax of expressions}
  \label{fig:syntax-expressions}
\end{figure}

% RUNTIME SYNTAX

\begin{figure}[t!]
  \begin{align*}
    \U,\V \grmeq
    & \C
      \grmor \EVar
      \grmor \EAbs \EVar\E
      \grmor \EInj{i} \V
      \grmor \EPair \U\V
      % \grmor \CSend \V 
              \tag{Values}
    \\
    \EC \grmeq&
              \ECEmpty
              \grmor \EApp \EC\E
              \grmor \EApp \V\EC
              \grmor \ESeq \EC\E
              \grmor \EPair \EC\E
              \grmor \EPair \V\EC
              \grmor \EInj{i}\EC
              \grmor \\
    & \ELet {\EPair \EVar\EVar}\EC\E
              \grmor 
      \ECase{\EC}{\EVar}{\E}{\EVar}{\E}
      % \EMatch{E}{\ell : \E_\ell}{\ell\in L}
              \tag{Evaluation contexts}
    \\
    \Bind \grmeq&
                  \EVar
                  \grmor \BEmp
                  \grmor \BSeq\Bind\Bind
                  \tag{Channel binders}
    \\
    \BindGroup \grmeq
    & \Bind
      \grmor \BPar\BindGroup\BindGroup
      \tag{Binder group}
    \\
    \Proc \grmeq
    & \PExp\E
      \grmor \PPar\Proc\Proc
      \grmor \PNew{\BindGroup}{\BindGroup}{\Proc}
      \tag{Processes}
    \\
    \FC \grmeq
    & \PExp \EC
      \tag{Thread evaluation contexts}
    \\
    \CC \grmeq
    & \ECEmpty
      \grmor \PPar{\CC}{\Proc}
      % \grmor \PNew \Bind\Bind \CC
      \tag{Process contexts}
  \end{align*}
  The operators on binders and binder groups are
  associative. Sequential composition $\BSeq{}{}$ comes with unit
  $\BEmp$. Moreover, $\BSeq{}{}$ and $\BPar{}{}$
  associate with one another as follows:
  we define $\BSeq{\Bind}{(\BPar{\BindGroup}{\BindGroup'})}$ as
  $\BPar{(\BSeq\Bind\BindGroup)}{\BindGroup'}$ inductively with base
  case $\BSeq{\Bind}{(\BPar{\Bind'}{\BindGroup})} =
  \BPar{(\BSeq\Bind\Bind')}{\BindGroup}$ with the same convention when
  sequencing from the right side.
  \caption{Run-time syntax}
  \label{fig:syntax-runtime}
\end{figure}

% EXP EVALUATION

\begin{figure}[t!]
  \begin{mathpar}
    \RuleExpRedApp \and
    \RuleExpRedSeq \and
    \RuleExpRedLet \and
    \RuleExpRedCaseLeft \and
    % \RuleExpRedCaseRight \and
    \RuleExpRedCtx
  \end{mathpar}
  \caption{Expression reduction ($\E \Reduces \E$)}
  \label{fig:expression-reduction}
\end{figure}

\begin{figure}[tp]
  \begin{mathpar}
    % \RuleProcCongRefl \and
    % \RuleProcCongTrans \and
    \RuleProcCongParComm \and
    \RuleProcCongParAssoc \and
    \RuleProcCongParUnit \and
    \RuleProcCongResSwap \and
    \RuleProcCongResComm \and
    \RuleProcCongExtend
  \end{mathpar}
  \caption{Process congruence}
  \label{fig:configuration-congruence}
\end{figure}

\begin{figure}[t!]
  \begin{mathpar}
    \RuleProcRedExp \and
    % \RuleProcRedLink \and
    \RuleProcRedNew \and
    \RuleProcRedFork \and
    \RuleProcRedCom \and
    \RuleProcRedSplit \and
    \RuleProcRedSplitLocal \and
    \RuleProcRedDrop \and
    \RuleProcRedAcquire \and
    \RuleProcRedClose \and
    % \RuleProcRedChoice \and
    \RuleProcRedPar \and
    \RuleProcRedBind \and
    \RuleProcRedStruct
  \end{mathpar}
  \caption{Process reduction ($\Proc \Reduces \Proc$)}
  \label{fig:process-reduction}
\end{figure}

\begin{figure}
  \begin{align*}
    \Mob ::=~
    & \MobMobile \grmor \MobStationary
    \tag{Mobility with $\MobMobile < \MobStationary$} \\
    % \O \grmeq & 0 \grmor 1 \grmor 2 \grmor \dots
    % \tag{Priorities} \\
    % \P, \Q \in {}& \Nat \cup \{ \bot, \top \}
    % \tag{Priority bounds} \\
    \Dir \grmeq
      & \DirLeft \grmor \DirRight \grmor \DirUnord
        % \grmor \DirNone
        \tag{Directions} \\
    \Eff \grmeq
      & \EffPure \grmor \EffImpure
        \tag{Effects with $\EffPure < \EffImpure$} \\
    \Pair \grmeq
      & \PairUnord \grmor \PairOrd
        \tag{Pairs} \\
        % \K ::=~
        % & \KType\Mul \grmor \KSession\Mul
        % \tag{Kinds} \\ % needed?
    \T \grmeq
      &
        \BUnit \grmor
        % \B \grmor % Let us use Unit as a representative of all base types
        \TArr{\Mob}{\Dir,\Eff}\T\T \grmor \T\Pair[\Dir]\T \grmor \TSum\T\T \grmor \S 
        \tag{Types} \\
    \S \grmeq
      & \SSkip \grmor \SSeq\S\S \grmor \STerm[\O] \grmor \SWait[\O] \grmor
        \SSend[\O]\T \grmor \SRecv[\O]\T
        \\
      &
        \grmor \SChoice[\O]{\ell:\S_\ell}{\ell\in L}
        \grmor \SBranch[\O]{\ell:\S_\ell}{\ell\in L}
    \\
    &
        \grmor \SRet \grmor \SAcq
      \grmor
      \SRec\SVar\S \grmor \SVar % no rec yet
        \tag{Session types}
    \\
    \Ctx \grmeq
      & \CtxEmpty \grmor \CtxVar \EVar\T \grmor \CtxExt\Ctx\Ctx \grmor \CtxExtUnord\Ctx\Ctx
        \tag{Typing environments}
    \\
    \CtxPatternRaw \grmeq
      & []
        \grmor \CtxExt{\CtxPatternRaw}\Ctx
        \grmor \CtxExt\Ctx{\CtxPatternRaw}
        \grmor \CtxExtUnord{\CtxPatternRaw}\Ctx
        \grmor \CtxExtUnord\Ctx{\CtxPatternRaw}
    \tag{Context patterns}
  \end{align*}
  % TODO: perhaps diff syntax for effects (not to be confounded with the
  % lin direction)
  Left context patterns
  \begin{mathpar}
    \inferrule{}{\IsLeftPat~[]}

    \inferrule{\IsLeftPat~\CtxPatternRaw}{\IsLeftPat~(\CtxExt{\CtxPatternRaw}\Ctx)}

    \inferrule{\IsLeftPat~\CtxPatternRaw}{\IsLeftPat~(\CtxExtUnord{\CtxPatternRaw}\Ctx)}

    \inferrule{\IsLeftPat~\CtxPatternRaw}{\IsLeftPat~(\CtxExtUnord\Ctx{\CtxPatternRaw})}
  \end{mathpar}

  Typing contexts and patterns are subject to the following laws
  \begin{mathpar}
    \CtxExt\CtxEmpty\Ctx = \Ctx \and
    \CtxExt\Ctx\CtxEmpty = \Ctx \and
    \CtxExtUnord\CtxEmpty\Ctx = \Ctx \and
    \CtxExtUnord\Ctx\CtxEmpty = \Ctx \\
    \CtxExt{\Ctx_1}{(\CtxExt{\Ctx_2}{\Ctx_3})} =
    \CtxExt{(\CtxExt{\Ctx_1}{\Ctx_2})}{\Ctx_3} \and
    \CtxExtUnord{\Ctx_1}{(\CtxExtUnord{\Ctx_2}{\Ctx_3})} =
    \CtxExtUnord{(\CtxExtUnord{\Ctx_1}{\Ctx_2})}{\Ctx_3} \and
    \CtxExtUnord{\Ctx_1}{\Ctx_2} = \CtxExtUnord{\Ctx_2}{\Ctx_1}
  \end{mathpar}
  Rewrite rule for weakening (apply transitively in any context modulo
  the monoid laws)
  \begin{mathpar}
    \CtxPar{(\CtxExt{\Ctx_1}{\Ctx_2})}{(\CtxExt{\Ctx_3}{\Ctx_4})}
    \preccurlyeq
    \CtxExt{(\CtxPar{\Ctx_1}{\Ctx_3})}{(\CtxPar{\Ctx_2}{\Ctx_4})}
  \end{mathpar}
  Laws for context-free session types
  \begin{mathpar}
    \SSeq\SSkip\S = \S \and
    \SSeq \S\SSkip = \S \and
    \SSeq{\S[1]}{(\SSeq{\S[2]}{\S[3]})} =
    \SSeq{(\SSeq{\S[1]}{\S[2]})}{{\S[3]}} \and
    \SRec\SVar\S = \S{[\SRec\SVar\S/\SVar]} \and
    \SSeq{(\SChoice[\O]{\ell:\S_\ell}{\ell\in L})}\S =
    \SChoice[\O]{\ell:\SSeq{\S_\ell}\S}{\ell\in L} \and
    \SSeq{(\SBranch[\O]{\ell:\S_\ell}{\ell\in L})}\S = \SBranch[\O]{\ell:\SSeq{\S_\ell}\S}{\ell\in L}
  \end{mathpar}
  \caption{Type and typing context syntax}
  \label{fig:syntax-types}
\end{figure}


% Unrestricted Types

% \begin{figure}
%   \begin{mathpar}
%     \RuleUBase \and
%     \RuleUProd \and
%     % \RuleUSum \and
%     \RuleUArr \and
%     \RuleUSkip \and
%     \RuleUSeq \and
%     \RuleUVar \and
%     \RuleURec
%   \end{mathpar}
%   PT: only $\DirNone$ arrows are unrestricted!
%   \caption{Unrestricted Types ($\Unr\T$)}
%   \label{fig:unrestricted-types}
% \end{figure}

% \begin{figure}
%   \begin{mathpar}
%     \RuleOProdOne \and
%     \RuleOProdTwo \and 
%     \RuleOArr 
%   \end{mathpar}
%   TODO: Should it be $\Ord\,T\PairOrd U$? Otherwise why to we have an
%   ordered-pair type constructor?\\
%   PT: (obsolete) I don't think there is a useful syntactical
%   classification for an ordered type and we get by with using $\Unr{}$.
%   \caption{Ordered Types ($\Ord\T$)}
%   \label{fig:ordered-types}
% \end{figure}

% Bounded Session Types

\begin{figure}
  \begin{mathpar}
    \RuleTeSkip \and
    \RuleTeSkipSeq \and
    \RuleTeSkipMu \\
    \RuleTeTerm \and
    \RuleTeRet \and
    \RuleTeSeqOne \and
    \RuleTeSeqTwo \and
    \RuleTeBranch \and
    \RuleTeChoice \and
    \RuleTeVar \and
    \RuleTeRec \\
    \RuleTeAcq \and
    \RuleTeBase \and
    \RuleTeProd \and
    \RuleTeVariant \and
    \RuleTeArr \and
    \RuleTeCtxEmpty \and
    \RuleTeCtxSeq \and
    \RuleTeCtxPar \and
    \RuleTeCtxVar
  \end{mathpar}
  A \emph{mobile} session type starts with an $\SAcq$ followed by a bounded
  session type. A mobile type or environment contains only mobile
  session types. A \emph{bounded} session type ends with a $\STerm$ or
  a $\SRet$, in the non-recursive case. In the recursive case, a
  session type is bounded, if all finite traces end (consistently) in
  $\STerm$ or $\SRet$. A \emph{skipping} session type is equivalent to
  a single $\SSkip$.

  % TODO: infinite types as well? \\
  % PT: yes! \\
  % PT: I think we need bounded function types where the introduction
  % ensures that only variables of bounded type are captured. These
  % functions can be argument to fork and these functions can be sent
  % over a higher-order channel. \\
  % PT: in \textsc{T-Fork}, $\Bounded{\Ctx}$ is used, so have to define
  % it for all types \\
  % VV: let's do fork as a typing rule \\
  % Remark: there are no unrestricted arrows anymore! However, we could
  % use the exclamation point modality from linear logic which cuts off
  % the context \\
  % PT: hmmm, the bounded predicate is not sufficient! \\
  % PT: suppose $\Ctx = \Ctx_1,\Ctx_2$. If $\Bounded{\Ctx_1}$, then
  % $\Ctx_2$ is really independent of $\Ctx_1$. Hence, some uses of
  % $\Unr{\Ctx_i}$ can be replaced by $\Bounded{\Ctx_i}$!
  \caption{Mobile types and contexts ($\Mobile\T$, $\Mobile\Ctx$);
    bounded session types ($\Bounded\S$); skipping session types ($\Skips\S$)}
  \label{fig:bounded-session-types}
\end{figure}

% Constant Typing

\begin{figure}
  \begin{mathpar}
    \RuleCTUnit \and
    \RuleCTNew \and
    % \RuleCTLink \and
    \RuleCTSplit \and
    \RuleCTSSplit \and
    \RuleCTDrop \and
    \RuleCTAcquire \and
    \RuleCTFork \and
    \RuleCTSend \and
    \RuleCTRecv \and
    \RuleCTTerm \and
    \RuleCTWait \and
    % \RuleCTSelect \and
  \end{mathpar}
  \caption{Constant Typing ($\HasCType\C\T$)}
  \label{fig:constant-typing}
\end{figure}


% Session Type Formation: no need
% Type Formation: no need

% Typing

\begin{figure}[tp]
  \begin{mathpar}
    \RuleTConst \and
    \RuleTVar \and
    % \RuleTAbsUnr \and
    \RuleTAbsLin \and
    \RuleTAbsLeft \and
    \RuleTAbsRight \and
    % \RuleTAppUnr \and
    \RuleTAppLin \and
    \RuleTAppLeft \and
    \RuleTAppRight \and
  \end{mathpar}
  \caption{Typing: constants, variables, and functions}
  \label{fig:typing-functinos}
\end{figure}
\begin{figure}
  \begin{mathpar}
    \RuleTPairUnord \and
    \RuleTPairOrd \and
    \RuleTLet \and
    \RuleTLetUnit \and
    \RuleTLetUnord \and
    \RuleTLetOrd \and
    \RuleTInLeft \and
    \RuleTCaseSum \and
    % \RuleTFork \and
    % \RuleTMatch \and
    % \RuleTMatchLin \and
    \RuleTWeakCE
    % \\
    % \Big(\RuleTWeakC
    % \and
    % \RuleTWeakE\Big)
  \end{mathpar}
   % TODO: Contraction? Exchange? Weakening on the right only?
  \caption{Typing ($\HasType\Ctx\E\T\Eff$)}
  \label{fig:typing}
\end{figure}

\begin{figure}[tp]
  \begin{mathpar}
    \RuleTExpr \and
    \RuleTPar \and
    \RuleTRes \\
    \RuleBindEmp \and
    \RuleBindSeq \and
    \RuleBindPar \and
    \RuleBindVar
  \end{mathpar}
  \caption{Process typing and binding}
  \label{fig:process-typing}
\end{figure}

% \begin{lemma}
%   If $\HasType \Ctx \E U \_$ 
%   and 
%   $\HasType {\CtxPattern {\CtxVar \EVar U}} {E[\EVar]} T \Eff$, then
%   $\HasType {\CtxPattern \Ctx} {E[\E]} T \Eff$
% \end{lemma}

% \begin{figure}[t!]
%   \begin{mathpar}
%     \RuleCSEmpty \and
%     \RuleCSSplit \and
%     \RuleCSLeft \and
%     \RuleCSRight \and
%     \RuleCSBoth
%   \end{mathpar}
%   (PJT: Is this needed?)
%   \caption{Typing context split}
%   \label{fig:context-split}
% \end{figure}



\end{document}

%%% Local Variables:
%%% mode: latex
%%% TeX-master: t
%%% End:
