% Congruence

\newcommand*\RuleProcCongRefl{
  \inferrule[C-Refl]{}{
    \Proc \Congruent \Proc
  }
}

\newcommand*\RuleProcCongParComm{
  \inferrule[C-ParComm]{}{
    \PPar{\Proc[1]}{\Proc[2]} \Congruent \PPar{\Proc[2]}{\Proc[1]}
  }
}

\newcommand*\RuleProcCongParAssoc{
  \inferrule[C-ParAssoc]{}{
    \PPar{\Proc[1]}{(\PPar{\Proc[2]}{\Proc[3]})} \Congruent \PPar{(\PPar {\Proc[1]}{\Proc[2]})}{\Proc[3]}
  }
}

\newcommand*\RuleProcCongTrans{
  \inferrule[C-Trans]{
    \Proc[1] \Congruent \Proc[2] \\
    \Proc[2] \Congruent \Proc[3]
  }{
    \Proc[1] \Congruent \Proc[3]
  }
}

\newcommand*\RuleProcCongParUnit{
  \inferrule[C-ParUnit]{}{
    \PPar{\PExp{\CUnit}}{\Proc} \Congruent \Proc
  }
}

\newcommand*\RuleProcCongResSwap{
  \inferrule[C-ResSwap]{}{
    \PNew{\BindGroup[1]}{\BindGroup[2]}\Proc \Congruent \PNew{\BindGroup[2]}{\BindGroup[1]}\Proc
  }
}

\newcommand*\RuleProcCongResComm{
  \inferrule[C-ResComm]{
    (\Fv(\BindGroup[1]) \cup \Fv(\BindGroup[2])) \cap (\Fv(\BindGroup[3]) \cup \Fv(\BindGroup[4])) = \emptyset
  }{
    \PNew{\BindGroup[1]}{\BindGroup[2]}{\PNew{\BindGroup[3]}{\BindGroup[4]}\Proc}
    \Congruent
    \PNew{\BindGroup[3]}{\BindGroup[4]}{\PNew{\BindGroup[1]}{\BindGroup[2]}\Proc} 
  }
}

\newcommand*\RuleProcCongExtend{
  \inferrule[C-Extend]{
    \Fv(\Proc[1]) \cap (\Fv(\BindGroup[1]) \cup \Fv(\BindGroup[2])) = \emptyset
  }{
    \PPar{\Proc[1]}{\PNew{\BindGroup[1]}{\BindGroup[2]}{\Proc[2]}}
    \Congruent
    {\PNew{\BindGroup[1]}{\BindGroup[2]}{(\PPar{\Proc[1]}{\Proc[2]})}}
  }
}




%%% Local Variables:
%%% mode: latex
%%% TeX-master: "../main"
%%% End:
